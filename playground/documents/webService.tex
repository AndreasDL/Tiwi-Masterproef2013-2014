\documentclass[11pt]{article}
\usepackage{a4wide}
\begin{document}
\title{Automated testbed Monitoring for jFed}
\author{Andreas De Lille}
\maketitle

\section{Inhoud}
Dit document gaat dieper in op het ontwerp van de webservice.
Zo zullen de verschillende mogelijke calls uitgelegd worden. Ook de interne werking van de webservice en de opbouw van de achterliggende database komen aan bod.
Het is de bedoeling dat ook het actualiseren van de monitoring informatie via deze service verloopt.

\section{Api Calls}
We kunnen de calls op delen in 2 grote groepen. 
\begin{itemize}
\item Results : Het opvragen van resultaten van een test.
\item Configuration : Het opvragen, aanpassen en aanmaken van testen en de bijhorende configuratie.
\end{itemize}

\section{Calls - Results}
Voor het opvragen van resultaten wordt gebruik gemaakt van een generische functie.

\subsection{Functies}
Er zijn een aantal functies voorzien voor het opvragen van de resultaten.
\begin{itemize}
\item last : Deze functie geeft het laatste/ de laatste resultaat/resultaten terug (per test,testbed).
\item list : Deze functie geeft een lijst resultaten die voldoen aan de opgegeven parameters. 
Zo kan deze lijst bijvoorbeeld beperkt worden tot een resultaten die overeenkomen met een bepaalde status of vallen tussen bepaalde data.
\item average : Deze functie geeft het gemiddelde van de resultaten (per test,testbed) tussen een periode.
\item detail : Deze functie geeft een detailweergave terug van een test. Deze detail weergave kan bijvoorbeeld ook logfiles bevatten om debugging te vergemakkelijken.
\end{itemize}

\subsection{Parameters}
Hier worden de parameters besproken. Deze parameters zullen het resultaat van de funties hierboven filteren.

\begin{itemize}
\item tests : Deze paramater duidt een of meerdere testen aan.
\item testbeds : Deze parameter duidt een of meerdere testbeds aan.
\item count : Deze parameter geeft het aantal laatste resultaten aan dat teruggegeven moet worden. 
Ook wanneer er enkel een till opgegeven is, worden de laatste resultaten voor de opgegeven tijd teruggegeven.
Wanneer er enkel een from parameter opgegeven is, dan worden de eerste x resultaten startend vanaf de from terug gegeven.
\item from : Geeft aan vanaf welke timestamp er gezocht moet worden.
\item till : Geeft aan tot welke timestamp de resultaten gezocht moeten worden.
\item status : Geeft aan welke status de resultaten moeten hebben, bijvoorbeeld Fail,Warn of Succes.
\item resultid : Geeft het id van een testresultaat waarvan men een detail weergave wil.
\item testname : De naam van de test. Dit is ENKEL nodig bij stitching tests, omdat deze over meerdere testbeds gaan. 
\end{itemize}
Indien bij tests of bij testbeds de waarde all wordt gegeven, worden respectievelijk alle tests of alle testbeds teruggegeven. Ook opgeven van lijsten is mogelijk.
\clearpage
\subsection{Voorbeelden}
Hieronder zal ik een aantal voorbeelden van calls en bijhorende antwoorden geven.
Voorlopig zijn deze voorbeelden enkel beperkt tot het opvragen van data met behulp van HTTP GET requests.
\subsubsection{last?testbed=urn-testbed0\&test=ping\&count=2}
Deze call geeft voor testbed0 de laatste 2 ping resultaten terug.
\begin{verbatim}
[{
    "testname": "ping",
    "testbedId": "urn-testbed0",
    "planId": 103,
    "resultId": 1499,
    "log": "http://www.urn-testbed0com/Logs/ping/log1499",
    "results": [{
        "name": "pingValue",
        "value": 39
      }],
    "timestamp": "2014-03-14 18:08:38"
  },{
    "testname": "ping",
    "testbedId": "urn-testbed0",
    "planId": 103,
    "resultId": 1500,
    "log": "http://www.urn-testbed0com/Logs/ping/log1500",
    "results": [{
        "name": "pingValue",
        "value": 35
      }],
    "timestamp": "2014-03-14 18:08:39"
}]
\end{verbatim}
\clearpage
\subsubsection{last?testbed=ALL\&test=ping}
Deze call geeft voor elk testbed de laatste ping resultaten terug.
\begin{verbatim}
[{
    "testname": "ping",
    "testbedId": "urn-testbed0",
    "planId": 103,
    "resultId": 1452,
    "log": "http://www.urn-testbed0com/Logs/ping/log1452",
    "results": [
      {
        "name": "pingValue",
        "value": 35
      }
    ],
    "timestamp": "2014-03-14 18:03:37"
  },{
    "testname": "ping",
    "testbedId": "urn-testbed1",
    "planId": 104,
    "resultId": 1453,
    "log": "http://www.urn-testbed1com/Logs/ping/log1453",
    "results": [
      {
        "name": "pingValue",
        "value": 56
      }],
    "timestamp": "2014-03-14 18:03:38"
  },{
    "testname": "ping",
    "testbedId": "urn-testbed2",
    "planId": 105,
    "resultId": 1454,
    "log": "http://www.urn-testbed2com/Logs/ping/log1454",
    "results": [{
        "name": "pingValue",
        "value": 39
      }],
    "timestamp": "2014-03-14 18:03:39"
}]
\end{verbatim}
\clearpage

\subsubsection{last?test=stitching\&testname=stitching-name2}
Deze functie geeft een detail resultaat terug van de stitchingtest met naam 'stitching-name2'.
Er kan eventueel een globaal resultaat toegevoegd worden om weer te geven of de stitching test in zijn geheel is gelukt.
\begin{verbatim}
{
  "testname": "stitching-name2",
  "testbeds": [
    "urn-testbed0",
    "urn-testbed1",
    "urn-testbed2"],
  "planId": 80,
  "resultId": 2360,
  "log": "http://www.stitching-name2com/Logs/stitching-name2/log2360",
  "results": [{
      "name": "setup",
      "value": "succes"
    },{
      "name": "getUserCredential",
      "value": "succes"
    },{
      "name": "generateRspec",
      "value": "succes"
    },{
      "name": "createSlice",
      "value": "succes"
    },{
      "name": "initStitching",
      "value": "succes"
    },{
      "name": "callSCS",
      "value": "succes"
    },{
      "name": "callCreateSlivers",
      "value": "succes"
    },{
      "name": "waitForAllReady",
      "value": "succes"
    },{
      "name": "loginAndPing",
      "value": "succes"
    },{
      "name": "callDeletes",
      "value": "succes"
    }],
  "timestamp": "2014-03-14 18:03:39"
}
\end{verbatim}
\clearpage

\subsubsection{Average?testbed=urn-testbed0\&test=ping\\ \&from=2014-03-1T18:03:37\\\&till=2014-03-11T18:03:37}
Deze call geeft het gemiddelde van alle waarden van de ping test op testbed0 terug.
\begin{verbatim}
{
  "testname": "ping-average",
  "testbedId": "urn-testbed0",
  "results": [
    {
      "name": "average-pingValue",
      "value": 38.666666666667
    }
  ],
  "from": "2014-03-1 18:03:37",
  "till": "2014-03-11 18:03:37"
}
\end{verbatim}
\subsubsection{list?testbed=urn-testbed0\&test=ping\\\&till="2014-03-14T18:08:39\\\&count=2}
Deze call geeft een lijst van de 2 laatste pingtests voor 2014-03-14 18:08:39 op testbed0 terug.
\begin{verbatim}
[{
    "testname": "ping",
    "testbedId": "urn-testbed0",
    "planId": 103,
    "resultId": 1499,
    "log": "http://www.urn-testbed0com/Logs/ping/log1499",
    "results": [{
        "name": "pingValue",
        "value": 39
      }],
    "timestamp": "2014-03-14 18:08:38"
  },{
    "testname": "ping",
    "testbedId": "urn-testbed0",
    "planId": 103,
    "resultId": 1500,
    "log": "http://www.urn-testbed0com/Logs/ping/log1500",
    "results": [{
        "name": "pingValue",
        "value": 35
      }],
    "timestamp": "2014-03-14 18:08:39"
}]
\end{verbatim}
\clearpage

\section{Calls - configuration}
Deze calls hebben het doel om de configuratie van een test op te vragen.

\subsection{Functies}
Er zijn (uiteraard) ook een aantal functies voor het opvragen van de testconfiguratie.
\begin{itemize}
\item TestDefinition : Geeft de definitie van een testweer. Deze bevat informatie over de naam, het commando. Verder duidt deze definitie ook aan welke parameters er nodig zijn en welke waarden er terug gegeven worden. Deze waarden zijn echter niet ingevuld. Het is de bedoel om hier bijvoorbeeld te defini\"eren dat een pingtest bijvoorbeeld bestaat uit een ping commando en dat deze een timeout en een testbed moet meekrijgen. Samengevat kan gesteld worden dat dit de beschrijving van een type test is.
\item TestInstance : Hierbij wordt een instance teruggegeven. Dit is de concrete invulling van een testDescription. Daarbij wordt de test gecombineerd met de ingevulde parameters. Zo zal hier ingevuld worden dat er op testbed2 om de 5 minuten een pingtest moet uitgevoerd worden.
\end{itemize}

Deze opsplitsing is noodzakelijk om flexibiliteit aan te bieden. 
Een ping test wordt op meerdere testbeds uitgevoerd. Door deze opbouw moet een ping test maar eenmalig gedefini\"eerd worden. Als we de vergelijking met mvc maken, geeft de interface van een klasse waar, de instance kan gezien worden als een object van die klasse.

\subsection{Parameters}

\begin{itemize}
\item test : Deze parameter duidt aan welke test opgevraagd wordt.
\item instance : Deze parameter duit aan welk instance opgevraagd wordt.
\end{itemize}
Het is ook mogelijk om als waarde ALL of een lijst van id's op te geven.

\subsection{Voorbeelden}
Hieronder worden een aantal mogelijke calls weergegeven. Voorlopig zijn deze voorbeelden beperkt tot het opvragen van data via http get requests.

\clearpage
\subsubsection{testDefinition?test=stitching}
Deze call geeft de beschrijving van een stitching test terug.
\begin{verbatim}
{ "definitionId": "stitching",
  "command": "stitch",
  "parameters": [{
      "name": "topology",
      "type": "string"
    },{
      "name": "testbeds",
      "type": "testbed[]"}],
  "return": [{
      "name": "setup",
      "type": "string",
      "description": "succes?"
    },{
      "name": "getUserCredential",
      "type": "string",
      "description": "succes?"
    },{
      "name": "generateRspec",
      "type": "string",
      "description": "succes?"
    },{
      "name": "createSlice",
      "type": "string",
      "description": "succes?"
    },{
      "name": "initStitching",
      "type": "string",
      "description": "succes?"
    },{
      "name": "callSCS",
      "type": "string",
      "description": "succes?"
    },{
      "name": "callCreateSlivers",
      "type": "string",
      "description": "succes?"
    },{
      "name": "waitForAllReady", "type": "string", "description": "succes?"
    },{
      "name": "loginAndPing", "type": "string", "description": "succes?"
    },{
      "name": "callDeletes", "type": "string","description": "succes?"
}]}
\end{verbatim}
\clearpage
\subsubsection{testDescription?test=all}
Hierbij worden alle testbeschrijvingen teruggegeven.
\begin{verbatim}
[{
    "definitionId": "ping",
    "command": "ping",
    "parameters": [{
        "name": "timeout", "type": "int"
      },{ 
        "name": "testbed", "type": "string"
    }],
    "return": [{
        "name": "pingValue", "type": "int", "description": "pingValue"
    	}]
    },{
    "definitionId": "stitching",
    "command": "stitch",
    "parameters": [{
        "name": "topology", "type": "string"
      },{
        "name": "testbeds","type": "testbed[]"
     }],
    "return": 
    [{
        "name": "setup", "type": "string",
        "description": "succes?"
      },{
        "name": "getUserCredential", "type": "string", "description": "succes?"
      },{
        "name": "generateRspec", "type": "string", "description": "succes?"
      },{
        "name": "createSlice", "type": "string", "description": "succes?"
      },{
        "name": "initStitching","type": "string", "description": "succes?"
      },{
        "name": "callSCS","type": "string","description": "succes?"
      },{
        "name": "callCreateSlivers","type": "string","description": "succes?"
      },{
        "name": "waitForAllReady", "type": "string", "description": "succes?"
      },{
        "name": "loginAndPing", "type": "string", "description": "succes?"
      },{
        "name": "callDeletes", "type": "string", "description": "succes?"
}]}]
\end{verbatim}
\clearpage
\subsubsection{testInstance?Instance=all}
Deze call geeft alle geplande tests terug.
\begin{verbatim}
[{
    "definitionId": "ping",
    "parameters": [{
        "name": "timeout", "value": 106
       },{
        "name": "testbed","value": "urn-testbed0"
    }],
    "frequency": 1488,
    "planId": 0
  },{
    "definitionId": "ping",
    "parameters": [{
        "name": "timeout","value": 136
      },{
        "name": "testbed","value": "urn-testbed1"
    }],
    "frequency": 575,
    "planId": 1
  },{
    "definitionId": "ping",
    "parameters": [{
        "name": "timeout","value": 106
      },{
        "name": "testbed","value": "urn-testbed2"
    }],
    "frequency": 1637,
    "planId": 2
  },{
    "definitionId": "stitching",
    "parameters": [{
        "name": "topology","value": "ring"
      },{
        "name": "testbeds",
        "value": [
          "urn-testbed0","urn-testbed3"
    ]}],
    "frequency": 9362,
    "planId": 3
  },
 \end{verbatim}
 \clearpage
 \begin{verbatim}
 {
    "definitionId": "stitching",
    "parameters": [{
        "name": "topology","value": "ring"
      },{
        "name": "testbeds",
        "value": [
          "urn-testbed0","urn-testbed3"
    ]}],
    "frequency": 6991,
    "planId": 4
  },{
    "definitionId": "stitching",
    "parameters": [{
        "name": "topology","value": "ring"
      },{
        "name": "testbeds",
        "value": [
          "urn-testbed0","urn-testbed3"
    ]}],
    "frequency": 8458,
    "planId": 5
}]
\end{verbatim}

\subsubsection{testInstance?Plan=2}
Deze call geeft de instance met id 2 terug.
\begin{verbatim}
{
  "definitionId": "ping",
  "parameters": [
    {
      "name": "timeout",
      "value": 108
    },
    {
      "name": "testbed",
      "value": "urn-testbed2"
    }
  ],
  "frequency": 3046,
  "planId": 2
}
\end{verbatim}

\clearpage
\section{Return waarden}
Er kan gekozen worden tussen xml en json. Het verschil tussen xml en json is dat xml een boomstructuur beschrijft. De json voorstel komt meer overeen met een hashmap. Xml is meer geschikt voor grote geavanceerde structuren. Aangezien het hier om eenvoudige monitoring informatie gaat, is het gebruik van xml af te raden. Verder dient te worden opgemerkt dat een xml-notatie van een object langer is dan de overeenkomstige json. Ook is het parsen van json eenvoudiger. Bijgevolg zal hier dus geen xml, maar json gebruikt worden. Door de mvc opbouw die ik hier zal gebruiken, zal het echter niet moeilijk zijn om xml functionaliteiten te voorzien. Dit kan achteraf eventueel toegevoegd worden.\\

Merk op dat de huidige situatie niet voorziet in een webservice. De website maakt rechtstreeks verbinding met de databank. Het tussenvoegen van een webservice kan overhead veroorzaken. Door calls te bundelen, zodat een call direct alle resultaten van een testbed teruggeeft, wordt deze overhead tot een minimum beperkt.\\

Sommige opvragen geven langere antwoorden terug. Zo is de status van een stitching test multivalued. Een stitching test zal proberen om connectie te maken met een testbed. Vervolgens zal hij aanmelden en een testnetwerk opzetten. Eenmaal het testnetwerk opgezet is zal hij proberen om te pingen tussen de verschillende nodes. Een stitching test kan bij elke stap mislopen. Daarom zal er in de status duidelijk vermeld moeten worden welke stappen gelukt zijn en welke niet. Zo kan het zijn dat een stitching test niet kan aanmelden, wat niet verwonderlijk is als hij bv. ook niet kan pingen naar een testbed. De status van een stitching test zal dan ook overeen komen met een lijst van statussen.\\

De teruggegeven data zal ook generiek gemaakt worden als een soort geneste hash/array waarin alle waarden zitten. Dit geeft het voordeel dat we ons geen zorgen moeten maken over attributen die niet ingevuld zijn, deze worden dan gewoon weggelaten. Het dan ook aan de client om de teruggegeven data om te zetten naar klassen, indien nodig.

\clearpage
\section{Database}
Hieronder wordt het ontwerp van de database uitgelegd.
In een eerste versie wordt het aantal tabellen nog beperkt om de uitwerking makkelijker en sneller te maken.
Later zullen 'hardcoded' lijsten met behulp van extra tabellen in de database zitten.
De eerste versie zal ook 2 tests bevatten de ping test met attributen testbed en timeout. De tweede test is de stitching test die een lijst van testbeds meekrijgt als parameter, samen met een opstelling zoals ring of line.

\subsection{Tabellen}
De database zal een postgresql database zijn en zal bestaan uit volgende tabellen.
\begin{itemize}
\item testbeds : Deze tabel zal informatie over de testbeds bijhouden
\item testDefinition : Deze tabel zal de testbeschrijving/testdefinitie bijhouden. Zo wordt hier bijvoorbeeld gedefini\"eerd dat een ping test bestaat uit een ping commando. Ook de parameters van een pingtest worden hier aangegeven. Zo zal een pingtest bijvoorbeeld een testbed moeten meekrijgen waar de pingtest op uitgevoerd moet worden en een timeout. Ook zal er gedefinieerd zijn dat deze test een waarde teruggeeft. Later kan er eventueel gedefinieerd worden wat er moet gebeuren met die waarde (bijvoorbeeld mailen indien een testbed offline blijkt te zijn).
Deze parameters zullen in de eerste versie nog 'hardcoded' in de tabel zitten (bv een array in json formaat). Later kan door gebruik te maken van een tabel dit gedeelte eventueel 'opgekuist' worden.
\item testInstance : Hierbij worden de instanties van de testdefinities bijgehouden. Deze tabel zorgt voor de link tussen de testdefinitie en de ingevulde parameters. In een eerste versie zitten deze parameters echter 'hardcoded' als een hash in json formaat opgeslagen. Later kan een extra tabel dit 'opkuisen'.
\item Results : Deze tabel zal de resultaten opslaan van de tests. Merk op dat indien de resultaten multivalued zijn (zoals de stitching test) er een array wordt opgeslagen. In een eerste versie zal dit zijn door de json van een array op te slaan.
\end{itemize}

Hieronder zal ik kort de (eerste) versie van de tabellen bespreken. De precieze aantal kolommen is momenteel nog niet vastgelegd. De nadruk bij de eerste versie ligt dan ook op de functionaliteit. De eerste versie moet aantonen dat het gekozen ontwerp flexibel genoeg is om aan de eisen te voldoen.
Merk ook de opsplitsing testDefinition <=> testInstance.

\subsubsection{Testbeds}
Deze tabel houdt de verschillende testbeds bij en bevat volgende kolommen :
\begin{itemize}
\item testbedId : Het id van het testbed. Aangezien elk testbed een uniek urn heeft zal dit waarschijnlijk gebruikt worden als testbedid.
\item name : de naam van het testbed.
\end{itemize}

\subsubsection{TestDefinition}
Deze tabel houdt de definities van de tests bij.
Hierbij wordt een test abstract omschreven als een commando met een aantal parameters en een of meerdere return waarden.
De tabel bevat volgende kolommen :
\begin{itemize}
\item testName : Dit is de naam van een test en moet tevens uniek zijn.
\item commando : Het commando (of script) dat uitgevoerd moet worden.
\item parameters : Een lijst van parameters die een commando nodig heeft. In de eerste versie is deze lijst opgeslagen als de json encodering van een array.
Voor elke parameter is er een naam , een type en een beschrijving.
\item return : Met return wordt een lijst weergegeven van de return waarden van een test. Bij een ping is dit simpelweg een ping waarde. Bij een stitching test is dit echter een lijst van resultaten van subtests. Ook dit wordt in een eerste versie eenvoudig opgeslagen. Voor elke waarde wordt een naam , een type en een beschrijving bijgehouden.
\end{itemize}

\subsubsection{testInstance}
Deze tabel zal later de testdefinition koppelen aan de tabel met parameters. In de eerste versie zal hij de waarden hardgecodeerd bevatten.
In deze tabel wordt er bijgehouden der er om de 5 minuten een ping test uitgevoerd moet worden op een testbedX met een timeout Y.
De tabel bevat volgende kolommen : 
\begin{itemize}
\item TestDefinition : Deze geeft weer om welk type test het gaat.
\item Frequency : De frequentie waarmee de test uitgevoerd moet worden.
\item Parameters : Een (in json gecodeerde) lijst van parameters.
\end{itemize}

\subsubsection{Results}
Deze tabel zal de resultaten bijhouden van de test.
Hij bevat volgende kolommen.
\begin{itemize}
\item testInstance : Het 'type' van de test
\item resultid : Later gebruikt om detail weergave op te vragen van een resultaat.
\item results : een (in json gecodeerde) lijst met resultaten. 
\item log : De link naar de link van de log file. Deze zal beschikbaar staan op het testbed, via de url is het mogelijk om hem op te vragen.
\item timestamp : Een tijdsaanduiding.
\end{itemize}

\subsection{Uitwerking - volgende stappen}
De uitwerking van deze webservice zal eerst bestaan uit het opvragen van testresultaten en testconfiguratie.
Na de aanpassingen in de database(uitwerken hoe de lijsten opgeslagen worden), zal het ook mogelijk moeten zijn om testen aan te passen en te wijzigingen. 
Verder moet het ook mogelijk zijn om de nieuwe resultaten op te slaan in de databank.


\end{document}