\newpage
\chapter{Keuzes}
In dit hoofdstuk worden belangrijke keuzes bij de uitwerking uitgelegd en gemotiveerd.

\section{Formaat Returnwaarden}
\npar
Er kan gekozen worden tussen xml en json. Het verschil tussen xml en json is dat xml een boomstructuur beschrijft. De json voorstel komt meer overeen met een hashmap. Xml is meer geschikt voor grote geavanceerde structuren. Aangezien het hier om eenvoudige monitoring informatie gaat, is het gebruik van xml af te raden. Bovendien is de xml-notatie van een object langer dan de overeenkomstige json-notatie. Ook is het parsen van json eenvoudiger. Bijgevolg zal hier dus geen xml, maar json gebruikt worden. Door de model-view-controller opbouw die gebruikt zal worden, kan xml-functionaliteit achteraf makkelijk toegevoegd worden. Dit is al voorzien door de parameter format, die standaard ingevuld wordt met json.
\npar

\section{Structuur returnwaarden}
De teruggegeven data zal ook generiek gemaakt worden als een soort geneste hash/array waarin alle waarden zitten. Dit geeft het voordeel dat we ons geen zorgen moeten maken over attributen die niet ingevuld zijn, deze worden dan gewoon weggelaten. Het dan ook aan de client om de teruggegeven data om te zetten naar klassen, indien nodig.

\section{Databank technologie}

\section{webservice technologie}