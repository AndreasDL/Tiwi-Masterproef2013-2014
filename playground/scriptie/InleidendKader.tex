\chapter{Inleidend kader}
{\samenvatting
In dit hoofdstuk wordt het achterliggende kader van de masterproef geschetst. Daarna wordt de opdracht uitgewerkt. De opdracht kan opgedeeld worden in drie delen. Het eerste deel is het maken van een API om de monitoringsinformatie beschikbaar te maken. Het tweede deel is het maken van een service die de monitoring uitvoert. Het derde en laatste deel is het maken van een loadtester die kijkt hoe een testbed reageert op een belasting.}

\section{FIRE}
\npar
Deze masterproef is een onderdeel van een groter Europees onderzoeksproject genaamd FIRE (Future Internet Research and Experimentation)\nomenclature{FIRE}{Future Internet Research and Experimentation}. FIRE is gericht is op onderzoek naar toekomstige internet- en netwerktechnologie\"en. Door onderzoekscentra te laten samenwerken\citep{Fire-what-is}, tracht FIRE het onderzoek vlotter te laten verlopen. FIRE heeft twee grote doelen: enerzijds de samenwerking tussen verschillende onderzoekscentra verbeteren, anderzijds het delen van testfacilities gemakkelijker maken.
\npar
Het eerste doel is de samenwerking tussen verschillende onderzoekscentra verbeteren. Onderzoekers binnen eenzelfde vakgebied komen vaak gelijkaardige problemen tegen. FIRE vermijdt dat men telkens het wiel opnieuw uitvindt, door deze onderzoekers gemakkelijker en meer te laten samenwerken. Hierdoor worden oplossingen en idee\"en meer gedeeld, zodat de ontwikkeling sneller kan verlopen.
\clearpage
\npar
Het tweede doel is het delen van testfaciliteiten gemakkelijker  maken. Door FIRE krijgt een onderzoeker van een onderzoekscentrum toegang tot testfaciliteiten van andere onderzoekscentra binnen FIRE. Testfaciliteit is een algemene term die duidt op zowel de hardware als de software die gebruikt wordt om testen te verrichten. Een concreet voorbeeld van een testfaciliteit is een testbed. Een testbed is een server of een verzameling servers waarop men experimenten laat lopen. 
\npar
Zo kunnen er op een testbed bijvoorbeeld een server en een aantal cli\"ents gesimuleerd worden. Deze worden verbonden met een aantal tussenliggende routers. Vervolgens wordt een videostream opgestart. Op deze videostream kan men storing introduceren door pakketten te droppen. De storing zal ervoor zorgen dat het beeld aan de client-side hapert. Om dit probleem op te lossen kunnen er technieken ingebouwd worden aan client-side. Zo kan er overgeschakeld worden naar een lagere kwaliteit indien blijkt dat de beschikbare bandbreedte onvoldoende is. Testen van degelijke technieken verlopen door gebruikt te maken van testbeds.

\section{GENI}
\npar
FIRE reikt echter verder dan Europa alleen, zo zijn er ook overeenkomsten met onderzoeksprojecten buiten Europa. Een voorbeeld daarvan is GENI (Global Environment for Network Innovations)\nomenclature{GENI}{Global Environment for Network Innovations}. Geni is een Amerikaans onderzoeksproject gericht om testfaciliteiten te bundelen en beschikbaar te stellen aan onderzoekers \citep{geni-what-is}. Door de samenwerking tussen beide projecten is het mogelijk dat onderzoekers geassocieerd met FIRE ook gebruik maken van testbeds binnen het GENI project en omgekeerd.

\section{Fed4FIRE}
\npar
Het probleem dat zich hierbij stelt is dat elk testbed een eigen interne werking heeft. Onderzoekers hebben nu wel toegang tot andere testbeds, maar moeten voor elk testbed eerst de specifieke configuratie leren. Verschillende testbeds laten samenwerken op deze manier is geen sinecure. Om dit eenvoudiger te maken, heeft men de federation architectuur ingevoerd. 
\npar
De invoering van deze architectuur, binnen FIRE is een onderdeel van het Fed4FIRE-project (Federation 4 FIRE)\nomenclature{FED4FIRE}{Federation 4 FIRE}. De federation architectuur die hier gebruikt wordt is SFA 2.0\nomenclature{SFA}{slice-based federation architecture}. De bedoeling is dat testbeds binnen FIRE een federatie vormen. Een eerste gevolg hiervan is dat alle onderzoekers, services en testbeds binnen FIRE vertrouwd worden. Hierdoor heeft een onderzoeker binnen FIRE toegang tot alle testbeds. Een tweede gevolg is dat de configuratie en interne werking van de testbeds gelijk zijn. SFA wordt in hoofdstuk \ref{SFA} besproken.

\section{jFed}
\npar
jFed werd door iMinds ontwikkeld \citep{iminds-jFed} en is een javatool die de SFA architectuur gebruikt om testbeds aan te sturen. iMinds is een onafhankelijk onderzoekscentrum dat opgericht werd door de Vlaamse overheid \citep{iMinds-what-is}. iMinds is leider van het FED4FIRE project \citep{iminds-FED4FIRE}.
\npar
Met behulp van jFed kunnen onderzoekers snel en eenvoudig netwerken simuleren en testen uitvoeren. Toch is er nog ruimte voor verbetering in jFed. Een van de voornaamste problemen is dat een onderzoeker niet weet of het testbed dat hij gebruikt betrouwbaar is. Bepalen of een vreemd gedrag in een experiment te wijten is aan eigen ontwikkelingen of aan het falen van een testbed, kan op deze manier zeer tijdrovend zijn.

\section{Opdracht: Monitoring van testbeds}
\npar
Om dit probleem op te lossen heeft iMinds een monitoringssysteem uitgebouwd\citep{fed4fire-second-fed-arch}. Merk op dat monitoring op meerdere niveau's mogelijk is zoals o.a. testbed,experiment,.. . De monitoring die hier besproken wordt, richt zich op het controleren van testbeds en kijkt of een testbed online is en hoeveel resources er beschikbaar zijn. 
\npar
Dit monitoringssysteem werkt, maar is door de snelle ontwikkeling niet voorzien op uitbereidingen. Bovendien is de informatie die hiermee verzameld wordt niet eenvoudig beschikbaar. De masterproef lost dit probleem op door een nieuw monitoringssysteem uit te bouwen. Dit systeem moet voorzien zijn op uitbreidingen zoals nieuwe soorten testen.
\npar
GENI heeft zelf ook een gedistribueerd monitoringssysteem uitgebouwd \citep{geni-monitor}. Deze service maakt gebruik van datastores \citep{geni-overview}. Een datastore houdt de monitoring informatie van een testbed of aggregate bij. Deze informatie wordt dan opgehaald door een collector. Naast het uitbouwen van een nieuw monitoringssysteem, moet ook onderzocht worden of het mogelijk is om het monitoringssysteem te integreren in dit groter monitoringsframework. De werking van de GENI monitor wordt later in de scriptie uitgebreid besproken.
\npar
Deze masterproef kan opgedeeld worden in drie stukken. \\
Het eerste deel is een API maken die de bestaande monitoringsinformatie beschikbaar maakt. Het is de bedoeling dat deze API een stevige basis vormt waarop andere applicaties kunnen gebouwd worden.
\npar
Het tweede deel is het maken van een monitoringsservice die de testbeds controleert. De data hiervan wordt dan via de API beschikbaar gemaakt. De monitoringsservice zal de huidige monitoringsservice vervangen.
\npar
Na uitwerking van eerste twee delen kan men eenvoudig kijken welke testbeds betrouwbaar zijn. Dit is echter niet voldoende. De testbeds hebben echter ook een educatief doel, namelijk het ondersteunen van labo's. De meeste docenten zijn echter niet altijd even overtuigd over het gebruik van testbed tijdens hun labo's. De voornaamste reden hiervoor is dat docenten geen garantie hebben dat het testbed de belasting, veroorzaakt door vele instructies van een grote groep, aankan. 
\npar
Het laatste en derde deel zal dit probleem oplossen door een systeem te voorzien om loadtesten uit te voeren. Bij een loadtest wordt een testbed belast, waarna de reactie van het testbed geanalyseerd wordt. De resultaten van deze analyse bepalen vervolgens of het labo met X personen vlot kan verlopen.