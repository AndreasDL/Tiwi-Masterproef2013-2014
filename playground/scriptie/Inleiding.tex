\newpage
\chapter{Inleiding}
\section{Bestaande situatie}
\npar
iMinds is een onafhankelijk onderzoekscentrum dat opgericht werd door de Vlaamse overheid.
Het is voornamelijk bezig met onderzoek omtrent ICT-innovatie. Bij dit onderzoek wordt veelvuldig gebruik gemaakt van testbeds. Een testbed is een verzameling van nodes waarmee netwerkopstellingen kunnen gesimuleerd worden. 
\npar
Een goed voorbeeld van een testbed is de Virtual Wall. De Virtual Wall is ontworpen voor het experimenteren met nieuwe netwerkoplossingen voor de volgende generatie van het Internet. De kracht van deze infrastructuur is dat ze de mogelijkheid biedt om op een eenvoudige wijze specifieke netwerkopstellingen op te zetten volgens de noden van het onderzoek. 
\npar
Een voorbeeld waarvoor dit testbed wordt gebruikt is dat men een server en een aantal cli\"ents definieert. Deze worden verbonden met een aantal tussenliggende routers. Vervolgens wordt een videostream opgestart. Op deze videostream kan men storing introduceren door pakketten te droppen. Deze storing zal ervoor zorgen dat het beeld aan de client-side hapert. Er kunnen technieken ingebouwd worden aan client-side om deze storing op te vangen. Zo kan er overgeschakeld worden naar een lagere kwaliteit indien blijkt dat de beschikbare bandbreedte onvoldoende is. Testen van degelijke technieken verloopt dan ook aan de hand van testbeds.
\npar
Onderzoekers bij iMinds hebben niet alleen toegang tot hun Virtual Wall, maar door samenwerkingsakkoorden met diverse onderzoekscentra hebben ze ook wereldwijd toegang tot andere en soms erg verschillende testbeds. Deze zijn elk apart ontwikkeld. Hierdoor maakt elk testbed gebruik van een eigen interface. Deze interface wordt aangeboden door middel van een federation API (application programming interface) \nomenclature{API}{application programming interface}. Een federation API is een interface die alle mogelijk functionaliteiten van een specifiek testbed aanbiedt. Hij staat in voor de communicatie tussen een programma en het testbed. Onderzoekers die met meerdere testbeds werken, moeten dan ook de werking van elk testbed apart bestuderen alvorens ermee aan de slag te kunnen. Ook rechtstreekse verbindingen leggen tussen verschillende testbeds wordt hierdoor moeilijker. 
\npar
Daarom werd in het Europese onderzoeksproject Fed4FIRE beslist om een aantal federation API's vast te leggen die gelijk moeten zijn op elk testbed. Dit maakt het mogelijk om tools te ontwikkelen die bruikbaar zijn op alle testbeds verenigd in dit project. Hiervoor werd de tool jFed ontwikkeld door iMinds. jFed biedt een wrapper aan rond de specifieke federation API's die in dat project worden toegepast. Daarenboven voorziet jFed in verschillende applicaties om enerzijds te kunnen testen of testbeds deze API's correct ondersteunen, en anderzijds om als onderzoeker op een eenvoudige wijze een experiment te kunnen opzetten. 
\npar
De kernmodule waar alle andere modules gebruik van maken is de "jFed library". De voornaamste functie van deze library is om de communicatie met de testbeds te vereenvoudigen. De hiervoor bruikbare testbeds implementeren enkele gestandaardiseerde API's, die de benodigde functionaliteiten aanbieden. De jFed library implementeert de client kant van deze API's (application programming interface), en voorziet Java interfaces voor de API-calls. De gebruiker zal hierdoor niet geconfronteerd worden met deze details. Zo zal de library o.a. de achterliggend SSL-connectie (secure sockets layer) \nomenclature{SSL}{secure sockets layer} en de bijhorende certificaten en private keys beheren. Verder zullen ook de nodige omzettingen gebeuren. Daarnaast bevat deze library vele hulpklassen en methoden om de API's aan te spreken.
\mijnfiguur{width=0.9\textwidth}{jFed}{Opbouw van jFed}
\npar
Een goed voorbeeld van een API die door jFed word ge\"implementeerd is de AM (Aggregate Manager) \nomenclature{AM}{Aggregate Manager} API, versie 2 en 3.  Een voorbeeld van een call, is de "CreateSliver" call in de AMv2 API.
De AMv2 API specificeert dat alle calls gedaan worden over een SSL verbinding met client-side certificaten. Verder specificeert de API de exacte argumenten die vereist zijn en het formaat van het resultaat. De jFed library biedt een Java methode "createSliver" in de klasse "AggregateManager2" aan. De argumenten worden opgegeven in een Java formaat. Zo zal bijvoorbeeld een datum-argument als een Java Date-object doorgegeven worden.  De jFed library zal de datum naar het RFC3339 formaat omzetten voor het naar de server verstuurd wordt. Ook het resultaat van de call wordt verwerkt door de jFed library terug omgezet.
\npar
Steunend op deze library zijn enkele tools zoals jFed probe en jFed compliance tester gebouwd. Deze hebben als doel het valideren of testbeds de federation API's correct ondersteunen. Tenslotte is er de jFed graphical user interface (GUI) \nomenclature{GUI}{graphical user interface} tool. Deze tool maakt het mogelijk om als eindgebruiker een experiment op een intu\"itieve manier op te zetten. De jFed GUI heeft momenteel echter een relatief complexe interface die gebruiksvriendelijker gemaakt kan worden. 

\section{Probleemstelling}
\npar
Er is echter nog ruimte voor verbetering in jFed. Zo worden er op geregelde tijdstippen verschillende testen automatisch uitgevoerd op testbeds. Deze testen geven o.a. weer of testbeds online zijn en of het aanmelden via een SSH-verbinding (secure shell)\nomenclature{SSH}{secure shell} gelukt is. Deze resultaten worden momenteel nog redelijk ad hoc bijgehouden. Men maakt geen gebruik van een databank, maar van bourne again shell (bash) \nomenclature{bash}{bourne again shell} scripts om de resultaten te archiveren. Resultaten op deze manier bijhouden, is uiteraard ineffici\"ent. Bovendien staat deze informatie momenteel enkel op een enkele specifieke website vermeld. 
\npar
Een onderzoeker wordt dus niet via zijn primaire gebruikersinterface op de hoogte gebracht van eventuele storingen met de testbeds opgenomen in zijn experiment. Dit maakt het voor onderzoekers soms moeilijk om te identificeren of een bepaald gedrag in hun experiment te wijten is aan de eigen ontwikkelingen, of aan de testbeds zelf. Hierdoor kan veel tijd onnodig verloren gaan tijdens de uitvoering en de analyse van het experiment. Een betere oplossing zou zijn dat bij het selecteren van een testbed de betrouwbaarheid weergegeven wordt. Zo kan een betrouwbaar testbed gekozen worden, om dergelijke situaties zoveel mogelijk te vermijden.
\npar
Niet enkel de betrouwbaarheid is van belang, ook het \quotes{comfort} is van belang. Een testbed met een snellere verbinding geniet de voorkeur op een testbed met een trage verbinding. Naast de betrouwbaarheid en comfort moet ook gekeken worden naar de vereisten. Indien een betrouwbaar testbed niet beschikt over voldoende resources kan het niet gebruikt worden voor een experiment.
\npar
Door rekening houden met de betrouwbaarheid, het comfort en de vereisten kan jFed de onderzoeker melden dat een gekozen testbed niet betrouwbaar is. Hiervoor is een monitoring service nodig die bijhoudt hoe vaak en hoelang een testbed offline is.