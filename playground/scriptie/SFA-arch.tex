\chapter{SFA-Architectuur}
\label{SFA}
{\samenvatting Deze masterproef is onderdeel van FIRE, een Europees onderzoeksproject naar een innovatief internet. Binnen FIRE maakt deze masterproef een monitoringsservice met een bijhorende API. FIRE gebruikt voor zijn testbedden de SFA architectuur. Deze architectuur is ontwikkeld om de configuratie en de aansturing van testbeds of aggregates over de hele wereld gelijk te maken. Hierdoor moeten onderzoekers maar \'e\'en configuratie leren, waarmee ze vervolgens op elk testbed kunnen werken. Dit hoofdstuk gaat dieper in op SFA.}

\section{Doel}
\npar
SFA (Slice-based Federation architecture)\nomenclature{SFA}{Slice-based Federation Architecture} is een framework dat gebruikt wordt om testbeds aan te sturen\citep{SFA-overview}. SFA is gebaseerd op FFA (First Federation Architecture)\nomenclature{FFA}{First Federation Architecture} en wordt gebruikt om \'e\'en van de FIRE doelstellingen, namelijk het delen van testbeds makkelijker maken, waar te maken. Doordat alle testbeds op een andere manier werken, is het voor een onderzoeker erg moeilijk om verschillende testbeds te gebruiken. Een onderzoeker moet eerst kennis maken met de specifieke configuratie van een testbed, alvorens hij ermee kon werken. Aangezien dit zeer tijdrovend is, is het noodzakelijk om de configuratie van testbeds overal gelijk te maken.
\npar
Een belangrijk begrip is een federation. Dit begrip heeft vooral te maken met authorisatie. Binnen een federatie worden testbeds, services en onderzoekers vertrouwd. De bedoeling van Fed4FIRE is dan ook de testbeds of aggregates binnen FIRE samen te voegen tot een federatie.
\clearpage
\npar
SFA is een standaard die door testbeds ge\"implementeerd wordt.
Hierdoor kan een onderzoeker die kennis heeft van SFA, direct ook werken met alle testbeds die er compatibel mee zijn. Zoals te zien is in Figuur \ref{customerservice} is de onderzoeker de klant en het testbed is de service provider.
\mijnfiguur{width=0.9\textwidth}{customerservice}{De onderzoeker (boven) is de klant, de testbeds (onderhouden door de onderste personen) is de service provider. Fed4FIRE voorziet de link tussen beide partijen.}
\npar
SFA bezit een aantal functionaliteiten. Zo is voorzien dat het beleid van de eigenaar nageleefd wordt. SFA voorziet in mechanismen om dit te controleren.
\npar
SFA moet ook voorzien dat operators onderhoud kunnen uitvoeren. Hiervoor moet het mogelijk zijn om machines te verwijderen of te vervangen. Ook toevoegen van nieuwe machines moet mogelijk zijn. 
\npar
Daarnaast moeten onderzoekers de mogelijkheid krijgen om experimenten aan te maken en de medewerkers voor het project te beheren. Hierbij hoort ook de autorisatie die gecontroleerd wordt door de eigenaars. Zo is het bijvoorbeeld mogelijk om maar een beperkt aantal mensen toegang te verlenen.
\clearpage
\section{Entiteiten}
\npar
SFA herkent 4 entiteiten.
\begin{enumerate}
\item Owners
\item Operators
\item Researchers
\item Identity anchors / identity providers
\end{enumerate}
Hierna volgt is een bespreking van elke entiteit met zijn verantwoordelijkheden.

\subsection{Owners}
\npar
De eigenaars of verantwoordelijken voor het testbed zijn verantwoordelijk voor de werking van zijn (deel van) het testbed. De Owners bepalen welke beleidsregels er van toepassing zijn. Deze beleidsregels worden aangeduid met SLA (Service Level Agreements)\nomenclature{SLA}{Service Level Agreements}. Figuur \ref{policies} geeft een voorbeeld van een aantal mogelijke beleidsregels. Zo kan het zijn dat de eigenaar van een testbed niet wil dat er commerci\"ele testen gebeuren zonder dat hij daarvan op de hoogte is (Figuur \ref{policies} Testbed 1). Een ander voorbeeld is dat de eigenaar uitgaande verbindingen beperkt (Figuur \ref{policies} Testbed 2).
\mijnfiguur{width=0.8\textwidth}{policies}{Eigenaars bepalen het beleid van hun testbed.}

\subsection{Operators}
\npar
Operators zorgen voor het onderhoud van het testbed. Dit onderhoud omvat o.a. herstellingswerken, beveiliging, voorkomen van schadelijke activiteiten.

\subsection{Researchers}
\npar
De onderzoeker is de klant. Hij gebruikt een testbed om er zijn experimenten op uit te voeren. Deze experimenten verlopen in het kader van zijn onderzoek.

\subsection{Identity providers}
\npar
Een identity provider of identity anchor is iemand die entiteiten rechten kan geven. Zo kan een identity provider een hoofdonderzoeker rechten geven om onderzoekers binnen zijn project te laten werken.

\section{Opbouw van een testbed}
Een testbed is opgebouwd uit meerdere onderdelen die hierna besproken worden.

\subsection{Component}
\npar
Een testbed bestaat uit vele onderdelen. Het primaire bouwblok van een testbed is een component. Een component is bijvoorbeeld een computer, router of switch. Indien de component een computer is, wordt deze ook een node genoemd. Een node is dus een computer, meestal binnen een testbed, verbonden met het netwerk. 
\subsection{Aggregate}
\npar
Al deze componenten worden gegroepeerd in aggregates. Een aggregate is een verzameling componenten die onder eenzelfde beheerder valt. Zo is de virtual wall2 van iMinds een aggregate omdat het beheer van dit volledige testbed onder iMinds valt.
\clearpage
\subsection{Aggregate manager}
\npar
Elk aggregaat wordt beheerd door een AM (aggregate manager). De aggregate manager is een stuk software dat een interface aanbiedt aan onderzoekers. Via deze interface kan bijvoorbeeld 'plaats' gereserveerd worden om een experiment op te zetten. Een aggregate manager vervult taken zoals het toewijzen van van slices (zie later) aan onderzoekers of aan een experimenten.
\subsection{Sliver}
\npar
Een component kan echter ook gemultiplexed worden zodat er meerdere experimenten tegelijk op kunnen draaien. Dit kan door bijvoorbeeld virtualisatie toe te passen. Het 'stuk' van een component wordt een sliver genoemd.
\subsection{Slice}
\npar
Een slice is een verzameling van slivers. Een slice is een abstract begrip dat omschreven kan worden als een container waarin een experiment draait. Vanuit het perspectief van de onderzoeker komt dit overeen met de testopstelling die hij ter beschikking heeft. Vanuit het perspectief van de owner of eigenaar van het testbed is dit een administratieve opdeling om bij te houden welke testen er waar gebeuren. Figuur \ref{slice} geeft een voorbeeld van een slice. Hierbij is duidelijk dat een slice over meerdere aggregates kan lopen.  Een slice kan over meerdere aggregates heen kan lopen. Deze aggregaten hebben elk een eigen beheerder en SLA (Service Level Agreements). Ook is op Figuur \ref{slice} duidelijk te zien dat een slice opgebouwd is uit slivers.
\mijnfiguur{width=0.9\textwidth}{slice}{Een slice (geel) bestaat uit een verzamling slivers (groen).}
\clearpage
\section{Communicatie met testbed via RSpec}
\npar
Een RSpec (Resource Specification) \nomenclature{RSpec}{Resource Specification}, is een XML file die een proefopstelling beschrijft\citep{geni-RSpec}. Het voordeel van dit formaat in plain tekst is dat onderzoekers zeer eenvoudig experimenten kunnen herhalen. Doordat een RSpec een volledige beschrijving is van een proefopstelling, is dit een meerwaarde in wetenschappelijke verslagen.
RSpecs kunnen opgedeeld worden in drie soorten. 
\npar
De eerste soort is een request RSpec. Een request RSpec beschrijft welke resources een onderzoeker wil gebruiken in zijn experiment. De AM (aggregate manager) antwoordt hierop met een manifest RSpec, zoals beschreven in Figuur \ref{RSpec} . Deze RSpec beschrijft de resources die gealloceerd zijn voor het experiment. Merk op dat dit proces transparant gebeurt, de meeste jFed tools zullen automatisch een RSpec genereren. Toch is het mogelijk om RSpecs zelf aan te maken, wat vooral voor complexere testopstellingen gebeurt.
\mijnfiguur{width=0.9\textwidth}{RSpec}{Een onderzoeker, of de tool die hij gebruikt stuurt eerst een request RSpec en krijgt vervolgens een manifest RSpec terug.}
\npar
De derde soort RSpec, de advertisement RSpec, wordt gebruikt bij de listResourcetest. De listResourcetest komt later in de scriptie aan bod. Een advertisement RSpec lijst alle resources op die beschikbaar zijn op een testbed.