\subsection{Databanken}
\npar
De huidige situatie voorziet niet in een centrale webservice \citep{FED4FIRE-doc}.
Wat er wel bestaat is een verzameling websites die rechtstreeks verbinding maken met een of meerdere achterliggende databanken. In de bestaande situatie zijn er 3 databanken:
\begin{enumerate}
\item flsmonitoring
\item flsmonitoring-international
\item scenarios
\end{enumerate}
\newpage
\subsubsection{Flsmonitoring en flsmonitoring-internation databank}
\npar
De eerste en de tweede databank bestaan uit een tabel waarin de laatste resultaten van elke test bijgehouden worden. Het verschil tussen deze 2 databanken komt overeen met de toegewezen categorie waarin het testbed zich bevind. De eerste databank bevat de lokale testbeds, de tweede bevat de internationale testbeds. Deze tabellen bevatten volgende kolommen:
\begin{itemize}
\item testbedid
\item testbedname
\item testbedurl
\item pinglatency
\item getversionstatus
\item aggregatetestbedstate
\item last-check
\end{itemize}
 De eerste 3 kolommen zijn duidelijk. \quotes{Pinglatency} houdt de waarde van de pingtest bij.
De kolommen \quotes{getversionsStatus} en \quotes{aggregatetestbedstate} worden gebruikt om de uitkomst van de getVersion test bij te houden. Deze test bevat o.a. het versie nummer van de aggregate manager (AM). Zoals eerder vermeldt, staat de AM in voor het omzetten van java-datatypes naar de overeenkomstige datatypes op het testbed. Verder zorgt de AM er ook voor dat, indien nodig, calls beveiligd worden. Doordat er geen ssl authenticatie nodig is om de versie van de AM op te vragen, wordt hij vaak gebruikt om de status van een server op te vragen. De kolom \quotes{last-check} bevat een timestamp om bij te houden waneer de lijn laatste werd aangepast.
\subsubsection{Scenario databank}
\npar
De laatste databank bestaat uit 3 tabellen. Het doel ervan is het bijhouden van informatie over de scenariotesten. Scenariotesten of stitchingtesten zijn complexe testen die uit meerdere subtesten bestaan. Eenvoudig gezegd zal een stitching test de verbinding tussen verschillende testbeds testen. Hiervoor worden op elk testbed meerdere resources aangevraagd. Deze zullen dan trachten naar elkaar te pingen. Indien een testbed offline is wordt de volledige test afgebroken. 
\clearpage
Hieronder staan de opeenvolgende stappen die een stitching of scenariotest doorloopt.
\begin{enumerate}
\item setUp
\item getUserCredential
\item generateRspec
\item createSlice
\item initStitching
\item callSCS
\item callCreateSlivers
\item waitForAllReady
\item loginAndPing
\item callDeletes
\end{enumerate}
De inhoud van elke subtest wordt buiten beschouwing gelaten.
Wat wel opgemerkt kan worden, is dat de tests opgedeeld worden in 3 groepen. Testen 1-6 zijn voorbereidende testen met als doel de configuratie van testen 7-9 in orde te brengen. Test 10 is de cleanup die de opgebouwde configuratie van stappen 1-6 terug ongedaan maakt.
Elke subtest heeft een resultaat. Een stitching test zou dus minstens 10 resultaten of statussen hebben. In de huidige versie zijn er slechts 3 statussen gedefini\"eerd. Indien alle subtesten gelukt zijn , is de stitching test is volledig gelukt. Wanneer enkel de voorbereiding is gelukt, maar stappen 7-9 niet of gedeeltelijk, zal de status gedeeltelijk gelukt aangeven.De laatste status geeft aan dat alle subtesten mislukt zijn, opzetten van de vereiste configuratie is dus ook mislukt.
\npar
De database is opgebouwd uit 3 tabellen.
\begin{itemize}
\item test-results
\item test-context
\item testbeds
\end{itemize}
De eerste resultaat houdt informatie bij over de testresultaten. De tweede tabel houdt de context van de test bij. De laatste tabel houdt de testbeds bij. De concrete invulling van de tabellen is van minder belang en wordt hier niet vermeld.
\clearpage
\subsection{Webpagina\rq s}
\npar
In de huidige versie zijn een aantal webinterfaces voorzien\citep{FED4FIRE-doc} .
Er is een webinterface die enkel de status van de lokale testbeds weergeeft. Daarnaast er een webinterface die de internationale testbeds weergeeft. 
Deze 2 webinterfaces geven de naam van het testbeds met de ping, een veld dat aanduid of de getVersion call gelukt is en het aantal beschikbare resources.
%\mijnfiguur{width=0.9\textwidth}{monitoringint}{Internationale testbed monitoring}
%\mijnfiguur{width=0.9\textwidth}{monitoringint2}{Lokale testbed monitoring}
\npar
\clearpage
Zowel de webinterfaces als het aanpassen van monitoringinformatie verloopt rechtstreeks via de databank. Er komt geen webservice bij kijken.
\mijnfiguur{width=0.9\textwidth}{monitoringOpbouw}{Geen webservice in de huidige versie}
\npar
De laatste webinterface geeft de resultaten van de scenariotesten weer. Er is ook de mogelijkheid om de log files van een test te bekijken. Tevens is het ook mogelijk om de geschiedenis van een scenariotest te bekijken.
