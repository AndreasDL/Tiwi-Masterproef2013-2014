\chapter{Voorbeeld loadtest}
\npar
Deze bijlag is een voorbeeld van een uitgevoerde loadtest als voorbereiding voor een practicum door de Griekse universiteit van Patras. Tijdens dit practicum zullen een 50-tal studenten elk 2 PC's gebruiken om TCP congestion testen. Hierbij zal de virtual wall of wall1 gebruikt worden als testbed. Doordat alle studenten hun experiment op zeer korte tijd van mekaar zullen starten, zal een grote belasting ontstaan op het testbed. Om te kijken of het testbed dergelijke belasting aankan, wordt een loadtest van 119 PC's uitgevoerd. Deze loadtest zal het practicum simuleren en weergeven of er problemen optreden. 
\npar
Figuur \ref{verslag119pcs} geeft een overzicht van de gebruikte computers. Deze computers zijn per twee opgedeeld in slices. Figuur \ref{verslagresources} geeft een overzicht van deze slices met bijhorende status. Op deze figuur komt een groen vak overeen met een slice die gelukt is, een rood vak daarentegen duidt op problemen.
\mijnfiguur{width=1\textwidth}{verslag119pcs}{Overzicht van de verschillende slices.}
\mijnfiguur{width=1\textwidth}{verslagresources}{Status van de slices.} 
\npar
Vermits er in Figuur \ref{verslagresources} maar een pc rood is, kan besloten worden dat het testbed dergelijke belasting kan verwerken. Figuur \ref{verslaggrafiek} hieronder geeft een grafiek van de belasting van het testbed op die dag. Uiterst rechts op de figuur zijn er 2 hoge waarden zichtbaar, deze zijn veroorzaakt door de stresstest. Doordat het testbed een dubbel aantal cores ziet dan dat er werkelijk zijn, moeten de percentages verdubbeld worden. De stresstest met 119 computers geeft bijgevolg een belasting van 80\% voor een bepaalde tijd.
\mijnfiguur{width=1\textwidth}{verslaggrafiek}{Belasting van het testbed, percentages moeten verdubbeld worden.} 
\clearpage
\npar
Vervolgens werd de test herhaald met 100 gebruikers die telkens 2 computers gebruiken. De veroorzaakte belasting is te zien in Figuur \ref{verslaggrafiek100users}. Ook hier moeten de percentages verdubbeld worden wat een belasting van 90\% geeft over een langere periode.
\mijnfiguur{width=1\textwidth}{verslaggrafiek100users}{Belasting van een testbed met 100 gebruikers} 
\npar
De test met 100 gebruikers bestaat eigenlijk uit 100 testen die tegelijk draaien. Voor elke test is bijgevolg een resultaat, dat te zien is op de monitoringsinterface\footnote{Deze monitoringsinterface is een kloon van de werkende monitoringsAPI die enkel stresstesten uitvoert.}. Figuur \ref{verslag100} geeft resultaten per test voor 100 users. Het is duidelijk dat er hier en daar problemen optreden, maar dat het overgrote deel wel slaagt.
\npar
Doordat de load veel hoger is dan wat de studenten in werkelijkheid zullen doen is de stresstest succesvol. 
\mijnfiguur{width=1\textwidth}{verslag100}{Weergave per test} 