\chapter{Inleiding}
\npar
Het gebruik van netwerken en het Internet om computers en allehande randapperatuur te verbinden zal in de toekomst alleen maar stijgen. Het is dan ook van groot belang dat onderzoek op dit gebied vlot en correct verloopt. Daarom is het aangewezen dat onderzoekers samenwerken om zo ideeën en nieuwe technologiën te delen. Daarnaast moeten er ook testfaciliteiten zijn om deze nieuwe technologiën te testen.
FIRE (Future Internet Research and Experimentation) is een Europees onderzoeksproject dat zich op deze doelen richt.
\npar
Om de configuratie en werking van de verschillende testbeds gelijk te trekken, is binnen FIRE de federation architectuur ontworpen. De invoering hiervan zit in het project FED4FIRE (Federation for FIRE). De federation architectuur die in deze masterproef behandeld wordt is de SFA 2.0 (Slice-federation-architecture). SFA deelt een testbed op in meerdere niveau's, het onderste niveau zijn de componenten. Een component is een computer, router, ... van een testbed. Omdat testbeds vaak door meerdere onderzoekers tegelijk gebruikt worden, worden componenten vaak gemultiplexed. Zo krijgt elke onderzoeker bijvoorbeeld 'een stuk' van een computer, een sliver. Meerdere slivers vormen samen een slice. Een slice is dus een verzameling componenten waarop een experiment gedraaid wordt. Daarnaast kan men de componenten ook groeperen in aggregaten. Een aggregaat is een verzameling componenten die valt onder eenzelfde beheerder. Een goed voorbeeld van een aggregate is de virtual wall. De virtual wall is een testfaciliteit van iMinds, gebruikt om netwerken te simuleren. De virtual wall is een aggregate, in die zin dat alle componenten beheerd worden door iMinds.
\npar
Het beheer van al deze verschillende aggregates is geen sinecure. Om dit beheer te vereenvoudigen heeft iMinds binnen het Fed4FIRE project een monitoringsservice gemaakt. Deze service is echter snel ontwikkeld en is niet geschikt voor uitbereidingen. Deze masterproef bestaat uit 2 grote delen. Het eerste deel is een nieuwe monitoringsservice maken die de testbeds controleert. De informatie die deze service verzameld zal beschikbaar gemaakt worden door een monitoringsAPI. 
\npar
FIRE werkt nog samen met een gelijkaardig project, GENI. GENI (Global Environment for Network Innovations) is een Amerikaans project met gelijkaardige doelstellingen als FIRE. Hierdoor is de samenwerking tussen beide projecten zeer hoog. Het tweede deel van de masterproef is de integratie van monitoringsAPI in het GENI project.
\npar
Hoofdstuk 1 vertelt eerst meer over de achtergrond van het project. Hier wordt ook kort de opdracht uitgelegd.
\npar
Hoofdstuk 2 gaat dieper in op de werking van de monitoringsservice die er was. Hierbij komt ook de probleemstelling uitgebreider aan bod.