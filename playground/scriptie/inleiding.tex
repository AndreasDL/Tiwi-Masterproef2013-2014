\chapter{Inleiding}
\npar
Het gebruik van netwerken en het internet om computers en allehande randapperatuur te verbinden zal in de toekomst alleen maar stijgen. Het is dan ook van groot belang dat onderzoek op dit gebied vlot en correct verloopt en dat onderzoekers samenwerken om zo idee\"en en nieuwe technologie\"en te delen. Daarnaast moeten er ook testfaciliteiten zijn om deze nieuwe technologie\"en te testen.
FIRE (Future Internet Research and Experimentation) is een Europees onderzoeksproject dat zich op deze doelen richt.
\npar
Om de configuratie en werking van de verschillende testbeds gelijk te trekken, is de federation architectuur ontworpen. De invoering hiervan zit in het FED4FIRE (Federation for FIRE) project. De federation architectuur die in deze masterproef behandeld wordt, is de SFA 2.0 (Slice-federation-architecture). SFA deelt een testbed op in meerdere niveau's, het onderste niveau zijn de componenten. Een component is een computer, router, ... van een testbed. Een term die hier ook vaak komt bij kijken is een node. Een node is een computer binnen een testbed. Een node is dus een component van een bepaald type. 
\npar
Omdat testbeds vaak door meerdere onderzoekers tegelijk gebruikt worden, worden componenten vaak gemultiplexed. Zo krijgt elke onderzoeker 'een stuk' van een computer, een sliver. Meerdere slivers vormen samen een slice. Een slice is dus een verzameling componenten waarop een experiment gedraaid wordt. 
\npar
Daarnaast kan men de componenten ook groeperen in aggregaten. Een aggregate is een verzameling componenten die valt onder eenzelfde beheerder. Een goed voorbeeld van een aggregate is de virtual wall. De virtual wall is een testfaciliteit van iMinds, gebruikt om netwerken te simuleren. De virtual wall is een aggregate, in die zin dat alle componenten beheerd worden door iMinds. SFA wordt later in de scriptie uitgebreid besproken.
\npar
Aggregates die een vertrouwensrelatie hebben vormen een federatie. Fed4FIRE zal er dus voor zorgen dat alle testbeds van FIRE een federatie vormen. Hierdoor worden onderzoekers binnen FIRE vertrouwd op alle testbeds binnen FIRE.
\clearpage
\npar
Het beheer van al deze verschillende aggregates is geen sinecure. Om dit beheer te vereenvoudigen heeft iMinds binnen het Fed4FIRE project een monitoringsservice gemaakt. Deze service is echter snel ontwikkeld en is niet geschikt voor uitbereidingen. Deze masterproef bestaat uit 2 grote delen. Het eerste deel is een nieuwe monitoringsservice maken die de testbeds controleert. De informatie die deze service verzamelt zal beschikbaar gemaakt worden door een monitoringsAPI. 
\npar
FIRE werkt samen met een gelijkaardig project, GENI. GENI (Global Environment for Network Innovations) is een Amerikaans project met gelijkaardige doelstellingen als FIRE. Hierdoor is de samenwerking tussen beide projecten zeer hoog. Het tweede deel van de masterproef is de integratie van monitoringsAPI in het GENI project.
\npar
Hoofdstuk 1 Situeert de masterproef. Hier wordt ook kort de opdracht uitgelegd.
\npar
Hoofdstuk 2 gaat dieper in op de SFA-architectuur. De SFA-architectuur wordt gebruikt om de configuratie en besturing van testbeds over heel de wereld gelijk te maken.
\npar
Hoofdstuk 3 maakt een analyse van de bestaande FIRE en GENI monitor en geeft hierbij ook de probleemstelling aan.
\npar
Hoofdstuk 4 geeft eerst de opbouw van de verschillende projecten binnen de masterproef. Vervolgens worden de gebruikte technologie\"en besproken.
\npar
Hoofdstuk 5 bespreekt per onderdeel de verschillende gebruikte technologie\"en.
\npar
Hoofdstuk 6 geeft een uiteenzetting van de werking van het monitoringssysteem.