\newpage
\chapter{Analyse}

\section{Bestaande uitwerking}
\npar
De ontwikkeling van de huidige situatie is door de sneller ontwikkeling, minder gestructureerd verlopen.
Hierdoor bestaat de software uit een basis versie gevolgd door een aantal \quotes{quick and dirty} toevoegingen. Er moet opgemerkt worden dat de huidige situatie werkt, maar het kan beter. Het gemist van structuur in opbouw zal op lange termijn leiden tot code die zeer moeilijk aan te passen is.
\npar
De oplossing van dit probleem en tevens het onderwerp van mijn masterproef is een webservice die de monitoring automatiseert. Hierbij komen een aantal vragen naar boven. Welke informatie moet bijgehouden worden? 
Wat bepaald de betrouwbaarheid van een testbed? Hoe nauwkeurig moet deze informatie bijgehouden worden? 

\subsection{Databanken}
\npar
De huidige situatie voorziet niet in een centrale webservice.
Wat er wel bestaat is een verzameling websites die rechtstreeks verbinding maken met een of meerdere databanken. Er zijn 3 databanken voorzien:
\begin{enumerate}
\item flsmonitoring
\item flsmonitoring-international
\item scenarios
\end{enumerate}
\newpage
\subsubsection{Flsmonitoring en flsmonitoring-internation databank}
\npar
In de eerste en de tweede databank bestaan uit een tabel waarin de laatste resultaten van elke test bijgehouden worden. Het verschil tussen deze 2 databanken komt overeen met de toegewezen categorie waarin het testbed zich bevind. De eerste databank bevat de lokale testbeds, de tweede bevat de internationale testbeds. Deze tabellen bevatten volgende kolommen:
\begin{itemize}
\item testbedid
\item testbedname
\item testbedurl
\item pinglatency
\item getversionstatus
\item aggregatetestbedstate
\item last-check
\end{itemize}
 De eerste 3 parameters zijn duidelijk. \quotes{Pinglatency} houdt de waarde van de pingtest bij.
De kolommen \quotes{getversionsStatus} en \quotes{aggregatetestbedstate} worden gebruikt om de uitkomst van de getVersion test bij te houden. Deze test bevat o.a. het versie nummer van de aggregate manager. Doordat er geen ssl authenticatie nodig is voor deze test, wordt hij vaak gebruikt om de status van een server op te vragen. De kolom \quotes{last-check} bevat een timestamp om bij te houden waneer de lijn laatste werd aangepast.
\subsubsection{Scenario databank}
\npar
De laatste databank bestaat uit 3 tabellen. Het doel ervan is het bijhouden van informatie over de scenariotesten. Scenariotesten of stitchingtesten zijn complexe testen die uit meerdere subtesten bestaan. Eenvoudig gezegd zal een stitching test de verbinding tussen verschillende testbeds testen. Hiervoor worden op elk testbed meerdere resources aangevraagd. Deze zullen dan trachten naar elkaar te pingen. Indien een testbed offline is wordt de volledige test afgebroken. Hieronder staan de opeenvolgende stappen die een stitching of scenariotest doorloopt.
\begin{enumerate}
\item setUp
\item getUserCredential
\item generateRspec
\item createSlice
\item initStitching
\item callSCS
\item callCreateSlivers
\item waitForAllReady
\item loginAndPing
\item callDeletes
\end{enumerate}
De inhoud van elke subtest wordt hier buiten beschouwing gelaten.
Wat wel opgemerkt kan worden is dat we de tests kunnen opdelen in 3 groepen. Zo zijn testen 1-6 voorbereidende testen. Ze dienen om de configuratie voor testen 7-9 klaar te zetten. Test 10 is de cleanup die de opgebouwde configuratie van stappen 1-6 terug ongedaan maakt.
Elke subtest heeft een resultaat. Een stitching test zou dus minstens 10 resultaten hebben. In de huidige versie zijn er slechts 3 statussen gedefini\"eerd. De stitching test is volledig gelukt, dit komt overeen met 10 geslaagde subtesten. De status is gedeeltelijk gelukt, dit komt overeen met de voorbereiding die wel gelukt is, maar de stappen 7-9 zijn niet of slechts gedeeltelijk gelukt. De laatste status geeft aan dat alle subtesten mislukt zijn.
\npar
De database is opgebouwd uit 3 tabellen.
\begin{itemize}
\item test-results
\item test-context
\item testbeds
\end{itemize}
De eerste resultaat houdt informatie bij over de testresultaten. De tweede tabel houdt de context van de test bij. De laatste tabel houdt de testbeds bij. De concrete invulling van de tabellen is van minder belang en wordt hier niet vermeld.
\clearpage
\subsection{Webpagina\rq s}
\npar
In de huidige versie zijn een aantal webinterfaces voorzien. 
Er is een webinterface die enkel de status van de lokale testbeds weergeeft. Daarnaast er een webinterface die de internationale testbeds weergeeft. 
Deze 2 webinterfaces geven de naam van het testbeds met de ping, een veld dat aanduid of de getVersion call gelukt is en het aantal beschikbare resources.
\npar
De laatste webinterface geeft de resultaten van de scenariotesten weer. Er is ook de mogelijkheid om de log files van een test te bekijken. Tevens is het ook mogelijk om de geschiedenis van een scenariotest te bekijken.
\section{Wat kan anders}
\subsection{Samenvoegen databases}
\npar
Een eerste situatie die beter kan is het bijhouden van de laatste resultaten.
Deze resultaten zitten in een databank, die voor elke testbed een lijn bevat. Nieuwe waarde overschrijven die lijn. Meteen kan opgemerkt worden dat het niet mogelijk is om statistieken over een lange termijn weer te geven.Het aanpassen van de monitoring waarden gebeurd door shell scripts. Deze worden periodiek uitgevoerd en lezen een configuratiefile in. Indien het testbed waarop de test uitgevoerd moet worden al in de databank zit is het id vermeld in de configuratiefile. Indien er geen id staat zal het script zelf een nieuwe lijn aanmaken en vervolgens de id wegschrijven naar de file. Eenmaal uitgevoerd, geeft de test een resultaat terug in de vorm van een xmlfile die door ge\"instaleerde commando\rq s geparset wordt. Vervolgens wordt de data weggeschreven naar de databank.
\npar
Door nieuwe resultaten toe te voegen op een nieuwe lijn kunnen we wel statistieken over langere termijn bijhouden.
Ook de configuratiefiles kunnen opnemen in de databank. Toevoegen van een nieuwe test is dan eenmalig een entry toevoegen aan de database.
Deze entry bevat dan het commando en de benodigde commando's.
\npar
Een tweede punt is de opdeling flsmonitoring en flsmonitoring-international. Deze 2 databanken zijn eigenlijk gelijk en kunnen gemakkelijk ondergebracht worden in een database.
Dit kan door een extra punt toe te voegen dat weergeeft in welke categorie het testbed zich bevind. Een andere mogelijkheid is de internationale testbeds hardcoderen op de monitoring site.
\npar
Door de databanken scenario\rq s en flsmonitoring in een databank onder te brengen, kunnen we het beheer vereenvoudigen.
Beide gegevens zijn eenmaal resultaten van testen. Deze meer generische aanpak zal er toe leiden dat testen eenvoudiger gedefini\"eerd kunnen worden.

\subsection{Structuur database}
\npar
De huidige opbouw van de databanken is niet flexibel genoeg. Als we de databanken zouden samenvoegen is er een complexere structuur nodig die meer flexibiliteit toelaat.
Het moet mogelijk zijn dat een tests meerdere argumenten meekrijgt. Dit aantal is verschillend per type tests, werken met extra kolommen is dus niet aangeraden.
Ook geven sommige testen meerdere resultaten terug. Dit aantal is ook variabel en afhankelijk van het type test dat gebruikt wordt.
De structuur van de databank wordt later in deze scriptie verder uitgewerkt.

\subsection{Webservice uitbouwen}
\npar
Doordat de webinterfaces rechtstreeks contact maken met de databank is er minder overhead. Als de databank echter ook vanuit jFed bereikbaar moet zijn, is het beter om een webservice tussen te voegen. Dit vermijd duplicatie van code. Deze service zal complexere zaken zoals filteren van resultaten voorzien. Daardoor kan zowel jFed als de webinterfaces en eventuele toekomstige toepassingen met een eenvoudige call de informatie ophalen en moet niet elke mogelijke applicatie terug zelf de filtering voorzien.
\npar
Eenmaal de webservice uitgebouwd is kunnen er sites gemaakt worden die deze webservice contacteren om informatie op te halen. De huidige webinterfaces moeten dus ook vervangen worden door nieuwe versies die wel gebruik maken van de webservice. Er kan ook integratie in jFed voorzien worden. Dit kan als een onderdeel van de masterproef, of als een latere uitbreiding voor jFed.

\section{Besluit}
\npar
Om aan de verwachten te voldoen zijn er 2 grote veranderingen nodig. 
Er zullen twee grote veranderingen gebeuren. De eerste eerst is het samenvoegen van de databanken. Om het beheer van de data te vereenvoudigen zou 1 grote databank het werk van de 3 kleinere databanken over nemen. Deze zou zo opgebouwd zijn dat we testen met een variabel aantal argumenten en resultaten kunnen toevoegen, zonder iets te wijzigen aan de structuur van de databank. Zowel de configuratie van de testen als de resultaten zouden in deze databank opgenomen zijn.