\newpage
\chapter{Analyse en probleemstelling}
{\samenvatting
Zoals eerder besproken, zal deze masterproef een monitoringsAPI maken in opdracht van iMinds, een onderzoekscentrum. De monitoringsAPI heeft als doel de resultaten van een achterliggende monitoringsservice aan te bieden. De monitorService zal op zijn beurt alle aggregaten (testfaciliteiten) binnen FIRE (Future Internet Research and Experimentation), een project voor de verbetering van onderzoek naar internet en netwerken, in de gaten houden. Dit hoofdstuk zal de werking van de FIRE monitor beschrijven. De FIRE monitor is de monitoringsservice die bij aanvang van de masterproef de monitoring verzorgde. Daarna zal er dieper ingegaan worden op de probleemstelling. Vervolgens wordt ook de GENI monitor kort besproken.}
\section{FIRE Monitor}
\npar
iMinds (onderzoekscentrum waar de masterproef uitgewerkt wordt en tevens leidinggevend in het FIRE project), heeft een monitoringsservice gemaakt die al enige tijd draait \citep{fed4fire-second-fed-arch}, de FIRE monitor. Deze monitoringsservice is ruimer dan de monitoring van testbeds, ook het monitoren van experimenten wordt door deze service afgehandeld\citep{fed4fire-second-fed-arch}. 
\npar
Deze monitoringsservice bestaat uit verschillende componenten. De eerste component is de Facility monitoring, deze monitoring wordt gebruikt bij de FLS (first level support). De first level support heeft als doel om de basis zaken te monitoring. De voornaamste test is de pingtest die kijkt of een testbed nog online is. De aggregate bepaald zelf welke testen er uitgevoerd worden.
\clearpage
\npar
De tweede component is de infrastucture monitoring. Deze component is gericht om componenten binnen een experiment. De verzamelde data bevat o.a. aantal verstuurde pakketten, aantal verloren pakketten, cpu-load, ... .
\npar
Een derde component is de OML measurement library, deze bibliotheek laat het toe dat een onderzoeker zijn eigen monitoring framework gebruikt om de metingen van zijn experiment te doen.
\npar
Deze masterproef richt zich op de Facility monitoring. De tweede en de derde component kunnen hier dus niet aanbod en worden verder buiten beschouwing gelaten. De monitoringsservice waarnaar verwezen wordt, is bijgevolg de Facility Monitoring.
\npar
Deze monitoringsservice (Facility Monitoring) is opgedeeld in een aantal stukken. Het eerste stuk is de FLS-monitor (First Level Support)\nomenclature{FLS}{First Level Support}. Deze is beschikbaar op https://flsmonitor.fed4fire.eu/,zie ook Figuur \ref{monitoringview} . Deze service heeft het doel actuele informatie weer te geven over de status van het testbed.
\mijnfiguur{width=0.9\textwidth}{monitoringview}{ testbed monitoring}
\npar
Figuur \ref{monitoringview} geeft een beeld van de monitoringssite. De laatste 2 kolommen zijn van minder belang. De eerste kolom geeft de naam van het testbed weer. Daarnaast wordt het resultaat van de laatste ping test getoond. De volgend 2 kolommen bevatten het resultaat van respectievelijk de getVersiontest en de free resources test.GetVersion geeft aan of de AM (aggregate manager) nog werkt terwijl de kolom free resources aangeeft hoeveel resources er nog beschikbaar zijn. De vorm van deze testen is relatief eenvoudig, een enkelvoudig resultaat wordt teruggegeven.

\clearpage
\npar
Het tweede deel van de monitoringsservice, nightly login testing, bevat complexere testen. Deze testen worden typisch 1 tot 2 keer per dag uitgevoerd. Deze testen zijn diepgaander dan de FLS-monitor. Zo wordt bij een logintest getest of het aanmelden op een testbed mogelijk is. Een andere test die uitgevoerd wordt, is de stitchingtest. Deze zal kijken of het mogelijk is om een netwerk op te zetten tussen verschillende testbeds. Zie Figuur \ref{monitoringStitch}
\mijnfiguur{width=0.94\textwidth}{monitoringStitch}{resultaten van de stitching test}
Zoals zichtbaar in Figuur \ref{monitoringStitchHist}, is het ook mogelijk om de geschiedenis van deze testen op te vragen. 
\mijnfiguur{width=1\textwidth}{monitoringStitchHist}{geschiedenis van resultaten}
\clearpage

\subsection{Probleemstelling}
\subsubsection{Bereikbaarheid}
\npar
Het voornaamste probleem is niet de structurering van de data, maar de bereikbaarheid. In de vorige versie is de data enkel bereikbaar via de webinterface, of rechtstreeks via de databank. Dit maakt het moeilijk voor nieuwe ontwikkelingen om deze data te gebruiken. Deze masterproef lost dit probleem op door het gebruik van een montoringsAPI. Deze zorgt ervoor dat de data vlot beschikbaar is via aanvragen aan de API.
\npar
Het ontwerpen van deze monitoringsAPI is het eerste deel van de masterproef. Deze API moet een vlotte toegang tot de resultaten garanderen. Ook moet het mogelijk zijn om deze resultaten de filteren. Een filter die er zeker noodzakelijk is, is het opvragen van de laatste resultaten van een testbed. Eenmaal deze api af is, kan alle communicatie met de achterliggende databank verlopen via de API. De website's die momenteel bestaat (zie Figuur \ref{monitoringView}), zal nu niet meer rechtstreeks contact maken met de databank. In plaats daarvan zal de API gebruikt worden als datalaag.
\npar
Ook toevoegen van nieuwe resultaten, door de monitoringsservice verloopt via de API. Dit heeft als voordeel dat de API complexere zaken zoals foutafhandeling kan afhandelen. Een bijkomend voordeel is dat de databank niet extern bereikbaar moet zijn. Authenticatie kan in een hogere laag afgehandeld worden. Merk op dat authenticatie geen onderdeel is van deze masterproef.

\subsubsection{Structuur}
\npar
Naast de bereikbaarheid is er ook een structureel probleem. De testen van de FLS monitor hebben een eenvoudige structuur. De testen zijn eenvoudig omdat ze slechts een beperkt aantal parameters nodig hebben en een enkelvoudig resultaat teruggeven. Zo geeft een pingtest de ping waarde terug. Een listResources test geeft het aantal beschikbare resource terug. Beide waarden zijn gewoon getallen, die opgeslagen worden in een kolom van de databank.
\npar
De nightly login testen zijn echter complexer. Deze bestaan uit meerdere opeenvolgende stappen, elke stap heeft een tussenresultaat. Hierdoor volstaat een kolom niet meer. De oplossing die toen aangewend is, is het gebruiken van een tweede databank. In die databank is er niet voor elk tussenresultaat een kolom voorzien, maar heeft men de resultaten samen genomen in 2 tussen resultaten. Deze manier lost het probleem van het variabele aantal tussenresultaten op, maar geeft geen proper overzicht van alle tussen statussen. Voor deze informatie moet men zich wenden tot de log.
\clearpage
\npar
Het is vanzelfsprekend dat deze manier van werken niet praktisch is. De monitoringsAPI moet immer alle tussenresultaten teruggeven zodat een vlot overzicht mogelijk is. Omdat het parsen van logfiles tijd in beslag neemt, moeten deze resultaten in de databank zitten. Om het gebruik van verschillende databanken te vermijden, is een nieuwe databank nodig die alle testen aankan. Samenvoegen van de databanken zorgt er immers voor dat gedeelde configuratie zoals testbeds en users (zie later), maar 1 keer moet worden opgeslagen.
\npar
Doordat de interface van een test sterk varieert, bezit de databank een complexere structuur. Hierbij is het aantal parameters en aantal tussenresultaten variabel. De uitwerking van deze structuur is het gebruik van een aparte tabel met daarin de tussenresultaten. Deze manier van werkt zorgt ervoor dat een variabel aantal resultaten opgeslagen kan worden.
\npar
De structuur van de databanken verbeteren is een tweede deel van de masterproef. De masterproef zal volledig nieuwe databank ontwerpen die alle vereisten aankan. Deze databank zal vervolgens geregeld worden via de monitoringsAPI.
\clearpage
\section{GENI monitor}
\npar
GENI (Global Environment for Network Innovations) is een amerikaans project dat , net zoals FIRE, een virtueel testlaboratorium heeft. Dit lab is gedeeld over meerdere onderzoekscentra. Aangezien GENI en FIRE beide bezig zijn met onderzoek naar innovatieve netwerk en internet ontwikkelingen, is de samenwerking tussen beide partijen vanzelfsprekend.
\npar
Testbed binnen GENI maken ook gebruikt van een Slice federation architectuur

\clearpage
\section{Besluit}
De vorige service werkte wel, maar was niet voorzien op de komst van complexere testen. Er zijn 2 grote problemen, enerzijds de bereikbaarheid van de data, anderzijds de structuur. Het eerste probleem is opgelost door het maken van een API. Het tweede probleem is opgelost door het uitbouwen van een complexe databank. Deze databank houdt zowel de resultaten als de configuratie van de testen bij.