\chapter{Situering}
{\samenvatting
In dit hoofdstuk wordt het achterliggende kader van de masterproef geschetst. Daarnaast wordt ook de opdracht uitgewerkt. De opdracht bestaat uit 2 grote delen enerzijds een monitoringsservice maken om testbeds te controleren. De informatie van deze monitoringsservice wordt via de monitoringsAPI beschikbaar gesteld aan de buitenwereld. Anderzijds wordt de monitoringsAPI geïntegreerd in een groter amerikaans framework.}
\section{Situering}
\npar
Deze masterproef is een onderdeel van een groter Europees onderzoeksproject genaamd FIRE (Future Internet Research and Experimentation)\nomenclature{FIRE}{Future Internet Research and Experimentation}. FIRE is een project dat erop gericht is om onderzoek naar toekomstige internet- en netwerktechnologiën te vergemakkelijken door onderzoekscentra te laten samenwerken\citep{Fire-what-is}. FIRE heeft twee grote doelen. Enerzijds de samenwerking tussen verschillende onderzoekscentra te verbeteren, anderzijds het delen van testfacilities makkelijker te maken.
\npar
Het eerste doel is de samenwerking tussen verschillende onderzoekscentra te verbeteren. Onderzoekers binnen eenzelfde vakgebied komen vaak gelijkaardige problemen tegen. FIRE vermijdt dat men telkens het wiel opnieuw uitvindt, door deze onderzoekers makkelijker en meer te laten samenwerken. Hierdoor worden oplossingen en ideeën meer gedeeld, zodat de ontwikkeling sneller kan verlopen.
\npar
Het tweede doel is het delen van testfaciliteiten makkelijker te maken. Door FIRE krijgt een onderzoeker van een onderzoekscentrum toegang tot testfaciliteiten van andere onderzoekscentra binnen FIRE. Testfaciliteit is een algemene term die duidt op zowel hardware als software dat gebruikt wordt om testen te verrichten. Een testbed is een concreet voorbeeld van een testfaciliteit. Een testbed kan gezien worden als een server of een verzameling servers waarop men testen kan laten lopen. Zo kan er op een testbed bijvoorbeeld een server en een aantal cli\"ents gesimuleerd worden. Deze worden verbonden met een aantal tussenliggende routers. Vervolgens wordt een videostream opgestart. Op deze videostream kan men storing introduceren door pakketten te droppen. Deze storing zal ervoor zorgen dat het beeld aan de client-side hapert. Er kunnen technieken ingebouwd worden aan client-side om deze storing op te vangen. Zo kan er overgeschakeld worden naar een lagere kwaliteit indien blijkt dat de beschikbare bandbreedte onvoldoende is. Testen van degelijke technieken verloopt dan ook aan de hand van testbeds.
\npar
Het probleem dat zich hier stelt is dat elk testbed op zijn eigen manier werkt. Onderzoekers hebben nu wel toegang tot andere testbeds, maar moeten voor elk testbed eerst de nieuwe configuratie leren. Verschillende testbeds laten samenwerken is, op deze manier, geen sinecure. Om deze configuratie gelijk te maken heeft men de federation architectuur ingevoerd. Dit is onderdeel van FED4FIRE (Federation 4 FIRE) \nomenclature{FED4FIRE}{Federation 4 FIRE}. De federation architectuur die hier gebruikt wordt is SFA 2.0\nomenclature{SFA}{slice-based federation architecture}. Deze architectuur heeft als doel om de configuratie en werking van alle testbeds gelijk te maken.
\npar
SFA 2.0 werkt met 3 niveau voor verantwoordelijkheid. De bovenste is de MA (Management Authority)\nomenclature{MA}{Management Authority}, deze is verantwoordelijk voor de stabiliteit van een heel testfaciliteit. Een SA (slice authority)\nomenclature{SA}{Slice Authority} is verantwoordelijk voor een of meerdere slices. De laatste is een gebruiker, bijvoorbeeld een onderzoeker die een experiment wil uitvoeren op een testbed.
\npar
Daarnaast maakt SFA ook gebruik van een specifieke naamgeving. Een component is een primaire block van de architectuur, bijvoorbeeld een computer of een router. Meerdere componenten worden vervolgens gegroepeerd in aggregaten. Alle componenten van een aggregaat vallen onder dezelfde MA (Management Authority). Elke aggregaat wordt gecontrolleerd door een AM (aggregate manager)\nomenclature{AM}{Aggregate Manager}. De AM beheert de allocatie van de verschillende experimenten op de aggregaat.
\npar
De MA (Management Authority) bepaald vervolgens hoe de resources verdeeld worden. Indien een component gemultiplexed wordt spreekt van men slivers. De gebruiker heeft dan een sliver van de component tot zijn beschikking. Meerdere slivers worden gegroepeerd tot aan slice. Een experiment wordt ook uitgevoerd binnen een slice.
\npar
jFed werd door iMinds ontwikkeld\citep{iminds-jFed} met als doel de SFA architectuur eenvoudig aan te sturen. 
iMinds is een onafhankelijk onderzoekscentrum dat opgericht werd door de Vlaamse overheid\citep{iMinds-what-is}. iMinds is leider van het FED4FIRE project\citep{iminds-FED4FIRE}.
Met behulp van jFed kunnen onderzoekers snel en eenvoudig netwerken simuleren en testen uitvoeren. Toch is er nog ruimte voor verbetering in jFed. Een van de voornaamste problemen is dat een onderzoeker niet weet of het testbed dat hij gebruikt betrouwbaar is. Bepalen of een vreemd gedrag in een experiment te wijten is aan eigen ontwikkelingen of aan het falen van een testbed, kan op deze manier zeer tijdrovend zijn.
\npar
Om dit probleem op te lossen heeft iMinds een monitoringssysteem uitgebouwd\citep{fed4fire-second-fed-arch}. Dit monitoringssysteem werkt, maar is door de snelle ontwikkeling niet voorzien op uitbereidingen. Deze masterproef zal enerzijds een monitoringsservice maken die deze testbeds in de gaten houdt. Anderzijds zal deze informatie via een monitoringsAPI beschikbaar gemaakt worden voor onderzoekers. Het is de bedoeling dat deze API een stevige basis vormt waarop andere applicaties kunnen gebouwd worden. Merk op dat monitoring op 3 niveau's mogelijk is: component,slice en aggregate. De monitoringsservice die hier besproken wordt, richt zich op de bovenste laag. De monitor zal dus kijken of een testbed online is en hoeveel vrije resources er beschikbaar zijn.
\npar
FIRE reikt echter verder dan Europa alleen, zo zijn er ook overeenkomsten met onderzoeksprojecten buiten Europa. 
FIRE werkt samen met GENI (Global Environment for Network Innovations)\nomenclature{GENI}{Global Environment for Network Innovations}. Geni is een Amerikaans onderzoeksproject gericht om aggregaten te bundelen en beschikbaar te stellen aan onderzoekers\citep{geni-what-is}.
FIRE en GENI werken ook aan de integratie van beide projecten\citep{fire-geni} \citep{geni-related}.
\npar
GENI heeft zelf een gedistribueerde monitoringservice uitgebouwd\citep{geni-monitor}. Deze service maakt gebruik van datastores\citep{geni-overview}. Een datastore houdt de monitoring informatie van een testbed of aggregate bij. Deze informatie wordt dan opgehaald door een collector. De webservice van FED4FIRE zou ook als een datastore bekeken worden. Op deze manier kan de monitoringsAPI geïntegreerd worden in een groter monitoringsframework. Dit zal als tweede deel van de masterproef behandeld worden.