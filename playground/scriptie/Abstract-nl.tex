\newpage
\chapter*{Abstract}
\npar
Door de huidige verschuiving naar cloudgebaseerde technologi\"en zal het belang 
van netwerkprotocollen en van de beschikbaarheid van netwerken alleen maar toenemen.
Om deze verschuiving vlot te laten verlopen is er meer en meer onderzoek nodig naar netwerktechnologi\"en.
Om onderzoekers en onderzoekscentra beter te laten samenwerken is FIRE (Future Internet Research and Experimentation) opgestart. Dit project zal trachten om de uitwisseling van ideeën tussen onderzoekers te verhogen. Daarnaast wordt, door de ontwikkeling van een gemeenschappelijke architectuur,  het delen van testfaciliteiten makkelijker gemaakt. Dit wordt verwezenlijkt via een java tool, jFed (java Federation)\nomenclature{jFed}{java Federation, tool voor het aansturen van de Federation architectuur}. jFed wordt gebruikt om testfaciliteiten binnen FIRE aan te sturen. Deze manier van werken zorgt ervoor dat men snel proefopstellingen kan maken op verschillende testfaciliteiten binnen FIRE. Toch zijn er enkele nadelen verbonden aan deze manier van werken. Het is voor een onderzoeker soms erg moeilijk om te bepalen of een bepaald gedrag in zijn experiment te wijten is aan eigen ontwikkelingen, of aan het falen van een testbed.
\npar
Deze masterproef verhelpt dit probleem door de invoer van een automatisch monitoringsproces. Een monitoringsservice zal instaan voor de monitoring van de testfaciliteiten. De informatie die hierbij verzameld wordt, wordt via een monitoringsAPI ter beschikking gesteld. Deze API (application programming interface)\nomenclature{API}{application programming interface} vormt een stevige basis waarop andere toepassingen kunnen gemaakt worden.
