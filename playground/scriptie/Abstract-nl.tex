\newpage
\chapter*{Abstract}
\npar
Door de huidige verschuiving naar cloudgebaseerde technologie\"en zal het belang 
van netwerkprotocollen en van de beschikbaarheid van netwerken alleen maar toenemen.
Om deze verschuiving vlot te laten verlopen is er meer onderzoek nodig naar netwerktechnologie\"en. Dit onderzoek kan beter verlopen als onderzoekers en onderzoekscentra beter samenwerken. Hiervoor is FIRE (Future Internet Research and Experimentation) opgestart. Dit samenwerkingsakkoord zal trachten om de uitwisseling van ideeën tussen onderzoekers te verhogen.
\npar
Daarnaast wordt, door de ontwikkeling van een gemeenschappelijke architectuur,  het delen van testfaciliteiten gemakkelijker gemaakt genaamd SFA (Slice Federation Architecture). Om het leven van onderzoekers gemakkelijker te maken is jFed ontworpen. jFed wordt gebruikt om testfaciliteiten aan te sturen via SFA. Deze manier van werken zorgt ervoor dat er snel proefopstellingen op verschillende testfaciliteiten. Toch zijn er ook enkele nadelen verbonden aan deze manier van werken. Het is voor een onderzoeker soms erg moeilijk om te bepalen of een bepaald gedrag in zijn experiment te wijten is aan eigen ontwikkelingen, of aan het falen van een testbed.
\npar
Deze masterproef verhelpt dit probleem door de invoer van een automatisch monitoringsproces. Een monitoringsservice zal instaan voor de monitoring van de testfaciliteiten. Hierbij wordt informatie verzameld die via een monitoringsAPI ter beschikking gesteld wordt. Deze API (application programming interface)\nomenclature{API}{application programming interface} vormt een stevige basis waarop andere toepassingen kunnen gemaakt worden.