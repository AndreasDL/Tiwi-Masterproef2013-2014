\newpage
\chapter*{Abstract}
\npar
Door de huidige verschuiving naar cloudgebaseerde technologi\"en zal het belang 
van netwerkprotocollen en beschikbaarheid van netwerken alleen maar toenemen.
Om deze verschuiving vlot te laten verlopen is er meer en meer onderzoek nodig naar netwerktechnologi\"en.
Voor dit onderzoek wordt er dan ook veelvuldig gebruik gemaakt van testbeds.
Testbeds worden aangestuurd via jFed en worden gebruikt om netwerken te simuleren.
De situatie heeft enkele nadelen. Het is voor een onderzoeker soms erg moeilijk om te bepalen of een bepaald gedrag in hun experiment te wijten is aan de eigen ontwikkelingen, of aan het testbed zelf.
\npar
Deze masterproef zal trachten dit probleem te verhelpen door de testbed monitoring te automatiseren. Via een webservice zal deze informatie dan aangeboden worden. Deze webservice zal weergeven hoe betrouwbaar een testbed is. Hiervoor komen verschillende ontwikkelingstools aan bod. Uiteindelijk zal deze service ervoor zorgen dat een onderzoeker via zijn primaire gebruikersinterface op de hoogte gebracht wordt van eventuele problemen.