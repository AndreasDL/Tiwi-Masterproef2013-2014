\newpage
\chapter*{Abstract}
\npar
Door de huidige verschuiving naar cloudgebaseerde technologie\"en zal het belang 
van netwerkprotocollen en van de beschikbaarheid van netwerken alleen maar toenemen.
Om deze verschuiving vlot te laten verlopen is er meer onderzoek nodig naar netwerktechnologie\"en. Dit onderzoek kan beter verlopen als onderzoekers en onderzoekscentra beter samenwerken. Hiervoor is FIRE (Future Internet Research and Experimentation) opgestart. FIRE is een Europees programma voor het uitvoeren van onderzoek naar het internet en de toekomst ervan. Hierbij werden universiteiten aangemoedigd om testbeds te bouwen.
\npar
Fed4FIRE is het sluitstuk van dit programma, waarbij al deze testbeds een universele API implementeren: SFA. Hierdoor wordt het delen van testfaciliteiten gemakkelijker gemaakt. Om te controleren of de SFA juist ge\"implementeerd is, is jFed ontworpen. 
jFed wordt periodiek gebruikt door beheerders om hun SFA-API implementatie te controleren.
\npar 
Deze masterproef bouwt hier verder op door een automatische monitoringsservice te maken. Deze kijkt periodiek of de SFA-API implementatie niet kapot is door nieuwe ontwikkelingen. Hierbij wordt informatie verzameld die via een monitoringsAPI ter beschikking gesteld wordt. Deze API (application programming interface)\nomenclature{API}{application programming interface} vormt een stevige basis waarvan andere toepassingen gebruik kunnen maken.
\npar
Binnen deze masterproef wordt de monitoringsservice ook uitgebreid met een loadtester. De loadtester wordt gebruikt om loadtesten uit te voeren. Hierbij wordt gekeken hoe een testbed reageert op een bepaalde belasting.