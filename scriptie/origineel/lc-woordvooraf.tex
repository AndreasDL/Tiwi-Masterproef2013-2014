\newpage
%\mbox{}\vspace{-1cm}
%\thispagestyle{plain}

%\textbf{\Huge{Woord vooraf}}

\chapter*{Woord vooraf}

%\vspace{2cm}

\begin{slshape}

%\small
Deze cursus werd in November 2003 voor het eerst uitgegeven in het kader van een Zeus (Studenten Werkgroep Informatica) initiatief om thesistudenten te helpen hun thesis in een aantrekkelijke vorm te gieten. De tekst is opgevat als een soort minithesis zodat studenten gemakkelijk alles kunnen overnemen. Om die reden zijn de bronbestanden beschikbaar gemaakt op het internet.\footnote{\url{http://zeus.ugent.be/~gaspard/latex}} 
\npar
Hoewel sommige deeltjes van deze cursus expliciet gericht zijn op het maken van een thesis, kan de tekst gebruikt worden als algemene inleiding op \latex. In 2004 zag ook Prof. Ottoy dit in. Vandaar dat deze cursus nu deel uitmaakt van de lessen informatica in de tweede bachelor bio-ir van de Universiteit Gent. Bij die gelegenheid heeft hij de cursus volledig nagekeken op taalfouten, waarvoor dank.
\npar
In 2006 heeft Prof. Dawyndt de cursus ook grondig herlezen en nagekeken, waarvoor dank. Sindsdien wordt deze cursus gebruikt in de lessen Computergebruik gegeven in de eerste bachelor Informatica van de Universiteit Gent.
\npar
Voor het schrijven van deze handleiding \latex, werd gebruik gemaakt van twee werken: \textsf{A guide to \latex} van \citet{kopka99} en \textsf{Handleiding \latex} van Piet van Oostrum (1996). Uit dit laatste werden zelfs hele paragrafen overgenomen. Daarnaast werd natuurlijk rijkelijk geput uit de documentatie die meegeleverd wordt met \latex zelf.
\npar
Minder belangrijke delen worden in een kleiner lettertype weergegeven. Verder zijn er verschillende voetnoten die verwijzen naar \latex documentatie op een Debian GNU/Linux systeem. Niemand gebruikt dat natuurlijk. Maar de bedoeling ervan is dat de lezer weet dat de documentatie bestaat, welke bestandsnaam die heeft en waar ongeveer die te vinden is in de \latex directorystructuur. Het is dan niet zo moeilijk meer om in je favoriete besturingssysteem te zoeken naar de desbetreffende bestandsnaam. 
\npar
In een standaard MikTeX installatie (d\'e \latex distributie voor Windows), zijn de helpfiles van de verschillende \latex packages te vinden in \bestand{C:\\texmf\\doc\\latex}. In \bestand{C:\\texmf\\doc\\guides} zijn enkele algemene handleidingen te vinden over \latex.
\npar
Het woord vooraf dient ook om mensen te bedanken: 

\begin{itemize}

\item De mensen van Zeus, voor de stimulerende Vrije--\engels{Open Source} sfeer en het organiseren van lessen hier rond. 

\item Schamper, het studentenblad van de Universiteit Gent, voor het leveren van de promotor van dit werk.

\item Rudy Gevaert, die in het academiejaar 2001-2002 als eerste een \latex les gaf.

\item Geert Vernaeve, voor het eerste contact met \latex en het C-voorbeeld op bladzijde \pageref{cvoorbeeld}.

\item Mensen die fouten rapporteerden en/of verbeteringen suggereerden: David De Wolf, Annelies Huyck, Yves Nevelsteen, Stijn Gors, Geert Vernaeve, Michiel Meire, Hendrik Maryns, Olivier Verhoogen, Jean-Pierre Ottoy, Hugo Coolens, Lieven Clement, Andy Peene, Reinout Debergh, Frederik De Schrijver, David van der Ha, Brecht Donckels, Dominique Lebbe, Christopher De Dobbelaere, Paul Vogels, Heidi Vanparys, Stijn Depuydt, Veerle Gevaert, Joke Van Hevele, Katrien De Dauw, Francis Santens, Nicolas Vanden Bossche en Peter Dawyndt.

\item De (thesis)studenten van het Boerenkot,
%\footnote{Ook nog wel Faculteit Landbouwkundige en Toegepaste Biologische Wetenschappen genoemd, maar niemand gebruikt die lange naam ;-)}
die in 2003 gevraagd hebben naar deze handleiding.

\end{itemize}

\latex lijkt in het begin moeilijk: alles in tekstmode, geen knopjes, je moet speciale commando's kennen om iets te bereiken~\ldots\ De eerste dagen zul je inderdaad enkele problemen ondervinden. Zoeken, doorbijten en hulp vragen aan meer ervaren gebruikers zullen ervoor zorgen dat je na enkele weken zelfs je wiskundige redeneringen rechtstreeks in \latex uitvoert. 
\npar
Deze handleiding is waarschijnlijk niet fautloos. Het rapporteren van meer dan ��n fout, zorgt voor je naam in de volgende editie van dit woord vooraf. Ook inhoudelijke opmerkingen zijn steeds welkom.

\vspace{4ex}

\hfill Gaspard Lequeux

\hfill Gent 9 Augustus 2006

%\normalsize
\end{slshape}

