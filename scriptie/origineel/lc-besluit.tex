
\chapter{Conclusies en perspectieven}

Elk goed werk eindigt met algemene slotconclusies (pleonasme) en een blik op wat beter kan (maar niet elk werk dat daarmee eindigt is een goed werk). Nu, bij deze cursus zijn er niet veel conclusies. Perspectieven zijn er des te meer. We hebben het vooral gehad over de syntax van \latex. Niet over de programma's er rond. Mensen die met \latex blijven werken, worden door de jaren heen zeer kritisch over de kwaliteit van hun documenten. Maar niet alleen hun documenten, heel hun (computer) tot op het pathologische af.\footnote{Waarom denk je, werd deze cursus geschreven op een Debian GNU/Linux systeem? Omdat dat gratis is? Nee! Omdat het beter is? Speelt vast en zeker mee. Maar de belangrijkste reden is de \textsc{Vrijheid} van dat systeem!} Bijgevolg zijn de programma's er rond zeer belangrijk.
\begin{itemize}
\item \command{make} Bij de finalisatie van een document, moeten verschillende commando's opgeroepen worden: 1 keer \command{latex}, 1 keer \command{bibtex}, 3 keer \command{latex}, 1 keer \command{makeindex}, 2 keer \command{latex}. Met een \bestand{Makefile} kan je dat allemaal inkorten tot ��n enkel commando: \command{make full}.
\item \command{ispell} Spellingscontrole.
\item \command{xfig} Voor maximale homogeniteit tussen de figuren en de tekst. De letters op je figuur zijn dezelfde als die in je document.
\item \command{vim} De teksteditor. In het begin moeilijk te gebruiken, maar ��n keer je er mee weg bent, lijken alle andere tekstverwerkers zo kleurloos.
\item \command{gimp} \engels{GNU Image Manipulation Program}. Om figuren op te poetsen.
\item \command{gnuplot}, \lcommand{grace} Voor het omzetten van data in grafieken.
\end{itemize}


