\chapter{Loadtest}
{\samenvatting Deze masterproef maakt een monitoringsservice met een bijhorende monitoringsAPI. Dit systeem zal testbeds monitoren en deze data toegankelijk maken om de betrouwbaarheid van een testbed te bepalen. Naast betrouwbaarheid is echter ook de performantie van belang. Door een aantal aanpassingen aan de monitoringsservice werd tijdens de uitwerking van de masterproef ook een tool gemaakt die loadtesten kan uitvoeren. Een loadtest is een test waarbij een bestaande testen binnen een kort tijdsinterval meerdere malen uitgevoerd wordt. De loadtest zal eerst een bestaande test ophalen en deze vervolgens op meerdere threads tegelijk starten.}
\section{Doel}
\npar
Een loadtest wordt gebruikt om te kijken welke belasting een testbed kan afhandelen. Een voorbeeld van een situatie waar dit nodig is, is wanneer een docent met een groep studenten het testbed wil gebruiken voor een labo. Hierbij zouden 50 studenten elk 2 computers gebruiken om tcp-congestie te testen. TCP-congestie wordt gebruikt om fileproblemen die ontstaan door het overlopen van buffers te verhelpen. Deze problemen ontstaan nadat een host meer informatie doorstuurt dan een tussenliggende router kan afhandelen. Hierdoor zal de buffer van de router overlopen, wat leidt tot verlies van pakketten. Aangezien het TCP protocol garandeert dat een overdracht zeker en compleet is, worden deze pakketten opnieuw verzonden. Als men echter de frequentie waarmee de pakketten verstuurd worden niet verlaagt, zal dit alleen maar leiden tot meer verloren pakketten.
\clearpage
\npar
Het probleem is dat de docent geen garantie heeft dat het testbed dit aankan. Hierdoor zijn vele leerkrachten weerhoudend om testbeds te gebruiken voor labo's.
Om aan te tonen dat het testbed wel een belasting van 50 studenten met elk 2 nodes aankan, wordt een stresstest uitgevoerd. Een voorbeeld van een uitgevoerde stresstest komt later in dit hoofdstuk aan bod. De bedoeling van deze stresstest is om 50 login testen tegelijkertijd uit te voeren. Deze zullen een belasting veroorzaken op het testbed die opgemeten en geanalyseerd wordt. Op basis van de resultaten kan de docent met een gerust hart gebruik maken van het testbed of net aangeraden worden om de groep op te delen.
\section{Uitwerking}
\npar
De uitwerking wordt weergegeven in Figuur \ref{structLoadtest}. Let op de overeenkomsten met Figuur \ref{structService}, waar de monitoringsservice wordt uitgelegd. De uitwerking van de loadtest is op een aantal stappen na, gelijk aan de uitwerking van de monitoringsservice, vermits de kern van beide gelijk is. 
\npar
Figuur \ref{structLoadtest} op volgende pagina geeft de uitwerking weer. Eerst wordt de bestaande test opgehaald. Daarna wordt een threadpool aangemaakt met de grootte n; het aantal testen dat uitgevoerd moet worden. Hierdoor kunnen alle testen tegelijkertijd uitgevoerd worden. Dit is niet mogelijk bij de service, vermits de grootte daar beperkt is. Vervolgens wordt elke test opgestart na een instelbaar interval. Dit interval zorgt er bijvoorbeeld voor dat de testen elkaar om de 2 seconden opvolgen.
\npar
Eenmaal een test is opgestart en uitgevoerd, wordt elk resultaat geparset en doorgegeven aan de resultuploader. Deze zal de resultaten \'e\'en voor \'e\'en uploaden en daarbij de uitvoeringstijd van de volgende test niet aanpassen. Dat laatste zou ertoe leiden dat volgende resultaten niet meer aanvaard worden. Merk het verschil met de monitoringsservice die hier wel een aanpassing doet van het nextrun veld, zie ook bijlage B.
\npar
Na het uitvoeren van de testen zijn de resultaten van de stresstest beschikbaar via de monitoringsAPI. Doordat er in de huidige implementatie geen onderscheid wordt gemaakt tussen resultaten van de monitoringsservice en de loadtesten wordt deze databank en API gekloond. De kloon hiervan zal vervolgens dienen om de resultaten van de loadtesten op te slaan. Samenvoegen van de monitoringresultaten met de loadtestresultaten is mogelijk, maar bleek noch een prioriteit, noch noodzakelijk te zijn.
\mijnfiguur{width=1\textwidth}{structLoadtest}{De uitwerking van de loadtesten.}
\clearpage
\section{Voorbeeld}
\npar
Deze sectie bevat een voorbeeld van een uitgevoerde loadtest als voorbereiding voor een practicum door de Griekse universiteit van Patras. Tijdens dit practicum zullen een 50-tal studenten elk 2 PC's gebruiken om TCP congestion testen. Hierbij zal de virtual wall of wall1 gebruikt worden als testbed. Doordat alle studenten hun experiment op zeer korte tijd van mekaar zullen starten, zal een grote belasting ontstaan op het testbed. Om te kijken of het testbed dergelijke belasting aankan, wordt een loadtest van 119 PC's uitgevoerd. Deze loadtest zal het practicum simuleren en weergeven of er problemen optreden. De loadtest is uitgevoerd met de code die geschreven werd in de masterproef. Het meten van resultaten en weergeven van grafieken is gebeurd door andere tools.
\npar
Figuur \ref{verslag119pcs} geeft een overzicht van de gebruikte computers. Deze computers zijn per twee opgedeeld in slices. Figuur \ref{verslagresources} geeft een overzicht van deze slices met bijhorende status. Op deze figuur komt een groen vak overeen met een slice die gelukt is, een rood vak daarentegen duidt op problemen.
\mijnfiguur{width=1\textwidth}{verslag119pcs}{Overzicht van de verschillende slices.}
\mijnfiguur{width=1\textwidth}{verslagresources}{Status van de slices.} 
\npar
Vermits er in Figuur \ref{verslagresources} maar \'e\'en pc rood is, kan besloten worden dat het testbed dergelijke belasting kan verwerken. Figuur \ref{verslaggrafiek} hieronder geeft een grafiek van de belasting van het testbed op die dag. Uiterst rechts op de figuur zijn er 2 hoge waarden zichtbaar, deze zijn veroorzaakt door de stresstest. Doordat het testbed een dubbel aantal cores ziet dan dat er werkelijk zijn, moeten de percentages verdubbeld worden. De stresstest met 119 computers geeft bijgevolg een belasting van 80\% voor een bepaalde tijd.
\mijnfiguur{width=1\textwidth}{verslaggrafiek}{Belasting van het testbed, percentages moeten verdubbeld worden.} 
\clearpage
\npar
Vervolgens werd de test herhaald met 100 gebruikers die telkens 2 computers gebruiken. De veroorzaakte belasting is te zien in Figuur \ref{verslaggrafiek100users}. Ook hier moeten de percentages verdubbeld worden wat een belasting van 90\% geeft over een langere periode.
\mijnfiguur{width=1\textwidth}{verslaggrafiek100users}{Belasting van een testbed met 100 gebruikers} 
\npar
De test met 100 gebruikers bestaat eigenlijk uit 100 testen die tegelijk draaien. Voor elke test is bijgevolg een resultaat, dat te zien is op de monitoringsinterface\footnote{Deze monitoringsinterface is een kloon van de werkende monitoringsAPI die enkel stresstesten uitvoert.}. Figuur \ref{verslag100} geeft resultaten per test voor 100 users. Het is duidelijk dat er hier en daar problemen optreden, maar dat het overgrote deel wel slaagt.
\npar
Doordat de load veel hoger is dan wat de studenten in werkelijkheid zullen doen is de stresstest succesvol. 
\mijnfiguur{width=1\textwidth}{verslag100}{Weergave per test} 