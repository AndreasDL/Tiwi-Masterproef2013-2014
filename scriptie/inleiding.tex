\chapter{Inleiding}
\npar
Het gebruik van netwerken en het internet om computers en allehande randapperatuur te verbinden zal in de toekomst alleen maar stijgen. Het is dan ook van groot belang dat onderzoek op dit gebied vlot en correct verloopt en dat onderzoekers samenwerken om zo idee\"en en nieuwe technologie\"en te delen. Daarnaast moeten er ook testfaciliteiten zijn om deze nieuwe technologie\"en te testen.
FIRE (Future Internet Research and Experimentation) is een Europees onderzoeksprogramma dat zich op deze doelen richt.
\npar
Om de configuratie en de werking van de verschillende testbeds gelijk te maken, is de federation architectuur ontworpen. De invoering hiervan zit in het Fed4FIRE (Federation for FIRE) project. De federation architectuur die in deze masterproef behandeld wordt, is SFA 2.0 (Slice-federation-architecture). Hierbij vormen alle testbeds van FIRE een federatie. Daardoor hebben onderzoekers binnen FIRE toegang tot alle testbeds binnen FIRE.
\npar
Het beheer van al deze verschillende testbeds is geen sinecure. Om dit beheer te vereenvoudigen is er binnen IBCN (Internet Based Communication Networks and Services), een onderdeel van het onderzoekscentrum iMinds, een monitoringsservice gemaakt. Deze service controleert periodiek of de SFA-API nog werkt. Echter door zijn snelle ontwikkeling is de service niet voorzien op uitbreidingen.
\npar
FIRE werkt samen met een gelijkaardig project, GENI. GENI (Global Environment for Network Innovations) is een Amerikaans project met gelijkaardige doelstellingen als FIRE. GENI is ook bezig met de ontwikkeling van een monitoringssysteem dat echter meer de nadruk legt op het monitoren van experimenten. De samenwerking tussen beide projecten kan bevorderd worden als beide monitoringssystemen compatibel zijn.
\npar
De opdracht van deze masterproef bestaat uit 3 grote delen. \\
Het eerste deel is het maken van een API die monitoringsdata beschikbaar maakt voor andere applicaties. Het tweede deel is een monitoringsservice maken die testbeds controleert. Het derde en laatste deel is de monitoringsservice uitbereiden om loadtesten uit te voeren. Met deze loadtesten wordt bekeken welke lading een testbed kan afhandelen.
\clearpage
\npar
De scriptie is al volgt uitgewerkt.
Hoofdstuk 1 Situeert de masterproef. Hier wordt ook kort de opdracht uitgelegd. In hoofdstuk 2 wordt de SFA-architectuur uitgelegd. Deze architectuur wordt gebruikt om de configuratie en besturing van testbeds over heel de wereld gelijk te maken.
\npar
Hoofdstuk 3 geeft meer uitleg over de bestaande FIRE en GENI monitor en geeft hierbij ook de probleemstelling aan. Hoofdstuk 4 geeft bespreekt het ontwerp van de masterproef. In hoofdstuk 5 wordt de implementatie van de verschillende onderdelen besproken.
\npar
Bijlage \ref{REFERENCE} bevat de referentie van de monitoringsAPI, geschreven in Engels.Bijlage \ref{ONDERHOUD}  bevat informatie voor de systeembeheerder.Bijlages \ref{REFERENCE} en \ref{ONDERHOUD} zijn specifiek gericht aan de systeembeheerder en eindgebruikers.